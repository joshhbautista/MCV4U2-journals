%%%%%%%%%%%%%%%%%%%%%%%%%%%%%%%%%%%%%%%%%
% The Legrand Orange Book
% LaTeX Template
% Version 2.4 (26/09/2018)
%
% This template was downloaded from:
% http://www.LaTeXTemplates.com
%
% Original author:
% Mathias Legrand (legrand.mathias@gmail.com) with modifications by:
% Vel (vel@latextemplates.com)
%
% License:
% CC BY-NC-SA 3.0 (http://creativecommons.org/licenses/by-nc-sa/3.0/)
%
% Compiling this template:
% This template uses biber for its bibliography and makeindex for its index.
% When you first open the template, compile it from the command line with the 
% commands below to make sure your LaTeX distribution is configured correctly:
%
% 1) pdflatex main
% 2) makeindex main.idx -s StyleInd.ist
% 3) biber main
% 4) pdflatex main x 2
%
% After this, when you wish to update the bibliography/index use the appropriate
% command above and make sure to compile with pdflatex several times 
% afterwards to propagate your changes to the document.
%
% This template also uses a number of packages which may need to be
% updated to the newest versions for the template to compile. It is strongly
% recommended you update your LaTeX distribution if you have any
% compilation errors.
%
% Important note:
% Chapter heading images should have a 2:1 width:height ratio,
% e.g. 920px width and 460px height.
%
%%%%%%%%%%%%%%%%%%%%%%%%%%%%%%%%%%%%%%%%%

%----------------------------------------------------------------------------------------
%	PACKAGES AND OTHER DOCUMENT CONFIGURATIONS
%----------------------------------------------------------------------------------------

\documentclass[12pt,fleqn]{book} % Default font size and left-justified equations

%%%%%%%%%%%%%%%%%%%%%%%%%%%%%%%%%%%%%%%%%
% The Legrand Orange Book
% Structural Definitions File
% Version 2.1 (26/09/2018)
%
% Original author:
% Mathias Legrand (legrand.mathias@gmail.com) with modifications by:
% Vel (vel@latextemplates.com)
% 
% This file was downloaded from:
% http://www.LaTeXTemplates.com
%
% License:
% CC BY-NC-SA 3.0 (http://creativecommons.org/licenses/by-nc-sa/3.0/)
%
%%%%%%%%%%%%%%%%%%%%%%%%%%%%%%%%%%%%%%%%%

%----------------------------------------------------------------------------------------
%	VARIOUS REQUIRED PACKAGES AND CONFIGURATIONS
%----------------------------------------------------------------------------------------

\usepackage{graphicx} % Required for including pictures
\graphicspath{{Pictures/}} % Specifies the directory where pictures are stored

\usepackage{lipsum} % Inserts dummy text

\usepackage{tikz} % Required for drawing custom shapes

\usepackage[english]{babel} % English language/hyphenation

\usepackage{enumitem} % Customize lists
\setlist{nolistsep} % Reduce spacing between bullet points and numbered lists

\usepackage{booktabs} % Required for nicer horizontal rules in tables

\usepackage{xcolor} % Required for specifying colors by name
\definecolor{ocre}{RGB}{243,102,25} % Define the orange color used for highlighting throughout the book


%----------------------------------------------------------------------------------------
%	MARGINS
%----------------------------------------------------------------------------------------

\usepackage{geometry} % Required for adjusting page dimensions and margins

\geometry{
	paper=a4paper, % Paper size, change to letterpaper for US letter size
	top=3cm, % Top margin
	bottom=3cm, % Bottom margin
	left=3cm, % Left margin
	right=3cm, % Right margin
	headheight=14pt, % Header height
	footskip=1.4cm, % Space from the bottom margin to the baseline of the footer
	headsep=10pt, % Space from the top margin to the baseline of the header
	%showframe, % Uncomment to show how the type block is set on the page
}

%----------------------------------------------------------------------------------------
%	FONTS
%----------------------------------------------------------------------------------------

\usepackage{avant} % Use the Avantgarde font for headings
%\usepackage{times} % Use the Times font for headings
\usepackage{mathptmx} % Use the Adobe Times Roman as the default text font together with math symbols from the Sym­bol, Chancery and Com­puter Modern fonts

\usepackage{microtype} % Slightly tweak font spacing for aesthetics
\usepackage[utf8]{inputenc} % Required for including letters with accents
\usepackage[T1]{fontenc} % Use 8-bit encoding that has 256 glyphs

%----------------------------------------------------------------------------------------
%	BIBLIOGRAPHY AND INDEX
%----------------------------------------------------------------------------------------

\usepackage[style=numeric,citestyle=numeric,sorting=nyt,sortcites=true,autopunct=true,babel=hyphen,hyperref=true,abbreviate=false,backref=true,backend=biber]{biblatex}
\addbibresource{bibliography.bib} % BibTeX bibliography file
\defbibheading{bibempty}{}

\usepackage{calc} % For simpler calculation - used for spacing the index letter headings correctly
\usepackage{makeidx} % Required to make an index
\makeindex % Tells LaTeX to create the files required for indexing

%----------------------------------------------------------------------------------------
%	MAIN TABLE OF CONTENTS
%----------------------------------------------------------------------------------------

\usepackage{titletoc} % Required for manipulating the table of contents

\contentsmargin{0cm} % Removes the default margin

% Part text styling (this is mostly taken care of in the PART HEADINGS section of this file)
\titlecontents{part}
	[0cm] % Left indentation
	{\addvspace{20pt}\bfseries} % Spacing and font options for parts
	{}
	{}
	{}

% Chapter text styling
\titlecontents{chapter}
	[1.25cm] % Left indentation
	{\addvspace{12pt}\large\sffamily\bfseries} % Spacing and font options for chapters
	{\color{ocre!60}\contentslabel[\Large\thecontentslabel]{1.25cm}\color{ocre}} % Formatting of numbered sections of this type
	{\color{ocre}} % Formatting of numberless sections of this type
	{\color{ocre!60}\normalsize\;\titlerule*[.5pc]{.}\;\thecontentspage} % Formatting of the filler to the right of the heading and the page number

% Section text styling
\titlecontents{section}
	[1.25cm] % Left indentation
	{\addvspace{3pt}\sffamily\bfseries} % Spacing and font options for sections
	{\contentslabel[\thecontentslabel]{1.25cm}} % Formatting of numbered sections of this type
	{} % Formatting of numberless sections of this type
	{\hfill\color{black}\thecontentspage} % Formatting of the filler to the right of the heading and the page number

% Subsection text styling
\titlecontents{subsection}
	[1.25cm] % Left indentation
	{\addvspace{1pt}\sffamily\small} % Spacing and font options for subsections
	{\contentslabel[\thecontentslabel]{1.25cm}} % Formatting of numbered sections of this type
	{} % Formatting of numberless sections of this type
	{\ \titlerule*[.5pc]{.}\;\thecontentspage} % Formatting of the filler to the right of the heading and the page number

% Figure text styling
\titlecontents{figure}
	[1.25cm] % Left indentation
	{\addvspace{1pt}\sffamily\small} % Spacing and font options for figures
	{\thecontentslabel\hspace*{1em}} % Formatting of numbered sections of this type
	{} % Formatting of numberless sections of this type
	{\ \titlerule*[.5pc]{.}\;\thecontentspage} % Formatting of the filler to the right of the heading and the page number

% Table text styling
\titlecontents{table}
	[1.25cm] % Left indentation
	{\addvspace{1pt}\sffamily\small} % Spacing and font options for tables
	{\thecontentslabel\hspace*{1em}} % Formatting of numbered sections of this type
	{} % Formatting of numberless sections of this type
	{\ \titlerule*[.5pc]{.}\;\thecontentspage} % Formatting of the filler to the right of the heading and the page number

%----------------------------------------------------------------------------------------
%	MINI TABLE OF CONTENTS IN PART HEADS
%----------------------------------------------------------------------------------------

% Chapter text styling
\titlecontents{lchapter}
	[0em] % Left indentation
	{\addvspace{15pt}\large\sffamily\bfseries} % Spacing and font options for chapters
	{\color{ocre}\contentslabel[\Large\thecontentslabel]{1.25cm}\color{ocre}} % Chapter number
	{}  
	{\color{ocre}\normalsize\sffamily\bfseries\;\titlerule*[.5pc]{.}\;\thecontentspage} % Page number

% Section text styling
\titlecontents{lsection}
	[0em] % Left indentation
	{\sffamily\small} % Spacing and font options for sections
	{\contentslabel[\thecontentslabel]{1.25cm}} % Section number
	{}
	{}

% Subsection text styling (note these aren't shown by default, display them by searchings this file for tocdepth and reading the commented text)
\titlecontents{lsubsection}
	[.5em] % Left indentation
	{\sffamily\footnotesize} % Spacing and font options for subsections
	{\contentslabel[\thecontentslabel]{1.25cm}}
	{}
	{}

%----------------------------------------------------------------------------------------
%	HEADERS AND FOOTERS
%----------------------------------------------------------------------------------------

\usepackage{fancyhdr} % Required for header and footer configuration

\pagestyle{fancy} % Enable the custom headers and footers

\renewcommand{\chaptermark}[1]{\markboth{\sffamily\normalsize\bfseries\chaptername\ \thechapter.\ #1}{}} % Styling for the current chapter in the header
\renewcommand{\sectionmark}[1]{\markright{\sffamily\normalsize\thesection\hspace{5pt}#1}{}} % Styling for the current section in the header

\fancyhf{} % Clear default headers and footers
\fancyhead[LE,RO]{\sffamily\normalsize\thepage} % Styling for the page number in the header
\fancyhead[LO]{\rightmark} % Print the nearest section name on the left side of odd pages
\fancyhead[RE]{\leftmark} % Print the current chapter name on the right side of even pages
%\fancyfoot[C]{\thepage} % Uncomment to include a footer

\renewcommand{\headrulewidth}{0.5pt} % Thickness of the rule under the header

\fancypagestyle{plain}{% Style for when a plain pagestyle is specified
	\fancyhead{}\renewcommand{\headrulewidth}{0pt}%
}

% Removes the header from odd empty pages at the end of chapters
\makeatletter
\renewcommand{\cleardoublepage}{
\clearpage\ifodd\c@page\else
\hbox{}
\vspace*{\fill}
\thispagestyle{empty}
\newpage
\fi}

%----------------------------------------------------------------------------------------
%	THEOREM STYLES
%----------------------------------------------------------------------------------------

\usepackage{tikz}
\newcommand\pacman[2]{\tikz[baseline, #1]{%
    \draw[thick,fill=#2]
    (0,0) -- (30:1cm) arc (30:330:1cm) -- cycle;
    \fill (0,2/3) circle (1.5mm);}
                       } 
                      

\usepackage{amsmath,amsfonts,amssymb,amsthm} % For math equations, theorems, symbols, etc

\newcommand{\intoo}[2]{\mathopen{]}#1\,;#2\mathclose{[}}
\newcommand{\ud}{\mathop{\mathrm{{}d}}\mathopen{}}
\newcommand{\intff}[2]{\mathopen{[}#1\,;#2\mathclose{]}}
\renewcommand{\qedsymbol}{$\blacksquare$}
\newtheorem{notation}{Notation}[chapter]

% Boxed/framed environments
\newtheoremstyle{ocrenumbox}% Theorem style name
{0pt}% Space above
{0pt}% Space below
{\normalfont}% Body font
{}% Indent amount
{\small\bf\sffamily\color{ocre}}% Theorem head font
{\;}% Punctuation after theorem head
{0.25em}% Space after theorem head
{\small\sffamily\color{ocre}\thmname{#1}\nobreakspace\thmnumber{\@ifnotempty{#1}{}\@upn{#2}}% Theorem text (e.g. Theorem 2.1)
\thmnote{\nobreakspace\the\thm@notefont\sffamily\bfseries\color{black}---\nobreakspace#3.}} % Optional theorem note

\newtheoremstyle{blacknumex}% Theorem style name
{5pt}% Space above
{5pt}% Space below
{\normalfont}% Body font
{} % Indent amount
{\small\bf\sffamily}% Theorem head font
{\;}% Punctuation after theorem head
{0.25em}% Space after theorem head
{\small\sffamily{\tiny\ensuremath{\blacksquare}}\nobreakspace\thmname{#1}\nobreakspace\thmnumber{\@ifnotempty{#1}{}\@upn{#2}}% Theorem text (e.g. Theorem 2.1)
\thmnote{\nobreakspace\the\thm@notefont\sffamily\bfseries---\nobreakspace#3.}}% Optional theorem note

\newtheoremstyle{blacknumbox} % Theorem style name
{0pt}% Space above
{0pt}% Space below
{\normalfont}% Body font
{}% Indent amount
{\small\bf\sffamily}% Theorem head font
{\;}% Punctuation after theorem head
{0.25em}% Space after theorem head
{\small\sffamily\thmname{#1}\nobreakspace\thmnumber{\@ifnotempty{#1}{}\@upn{#2}}% Theorem text (e.g. Theorem 2.1)
\thmnote{\nobreakspace\the\thm@notefont\sffamily\bfseries---\nobreakspace#3.}}% Optional theorem note

% Non-boxed/non-framed environments
\newtheoremstyle{ocrenum}% Theorem style name
{5pt}% Space above
{5pt}% Space below
{\normalfont}% Body font
{}% Indent amount
{\small\bf\sffamily\color{ocre}}% Theorem head font
{\;}% Punctuation after theorem head
{0.25em}% Space after theorem head
{\small\sffamily\color{ocre}\thmname{#1}\nobreakspace\thmnumber{\@ifnotempty{#1}{}\@upn{#2}}% Theorem text (e.g. Theorem 2.1)
\thmnote{\nobreakspace\the\thm@notefont\sffamily\bfseries\color{black}---\nobreakspace#3.}} % Optional theorem note
\makeatother

% Defines the theorem text style for each type of theorem to one of the three styles above
\newcounter{dummy} 
\numberwithin{dummy}{section}
\theoremstyle{ocrenumbox}
\newtheorem{theoremeT}[dummy]{Theorem}
\newtheorem{problem}{Problem}[chapter]
\newtheorem{exerciseT}{Exercise}[chapter]
\theoremstyle{blacknumex}
\newtheorem{exampleT}{Example}[chapter]
\theoremstyle{blacknumbox}
\newtheorem{vocabulary}{Vocabulary}[chapter]
\newtheorem{definitionT}{Definition}[section]
\newtheorem{corollaryT}[dummy]{Corollary}
\theoremstyle{ocrenum}
\newtheorem{proposition}[dummy]{Proposition}

%----------------------------------------------------------------------------------------
%	DEFINITION OF COLORED BOXES
%----------------------------------------------------------------------------------------

\RequirePackage[framemethod=default]{mdframed} % Required for creating the theorem, definition, exercise and corollary boxes

% Theorem box
\newmdenv[skipabove=7pt,
skipbelow=7pt,
backgroundcolor=black!5,
linecolor=ocre,
innerleftmargin=5pt,
innerrightmargin=5pt,
innertopmargin=5pt,
leftmargin=0cm,
rightmargin=0cm,
innerbottommargin=5pt]{tBox}

% Exercise box	  
\newmdenv[skipabove=7pt,
skipbelow=7pt,
rightline=false,
leftline=true,
topline=false,
bottomline=false,
backgroundcolor=ocre!10,
linecolor=ocre,
innerleftmargin=5pt,
innerrightmargin=5pt,
innertopmargin=5pt,
innerbottommargin=5pt,
leftmargin=0cm,
rightmargin=0cm,
linewidth=4pt]{eBox}	

% Definition box
\newmdenv[skipabove=7pt,
skipbelow=7pt,
rightline=false,
leftline=true,
topline=false,
bottomline=false,
linecolor=ocre,
innerleftmargin=5pt,
innerrightmargin=5pt,
innertopmargin=0pt,
leftmargin=0cm,
rightmargin=0cm,
linewidth=4pt,
innerbottommargin=0pt]{dBox}	

% Corollary box
\newmdenv[skipabove=7pt,
skipbelow=7pt,
rightline=false,
leftline=true,
topline=false,
bottomline=false,
linecolor=gray,
backgroundcolor=black!5,
innerleftmargin=5pt,
innerrightmargin=5pt,
innertopmargin=5pt,
leftmargin=0cm,
rightmargin=0cm,
linewidth=4pt,
innerbottommargin=5pt]{cBox}

% Creates an environment for each type of theorem and assigns it a theorem text style from the "Theorem Styles" section above and a colored box from above
\newenvironment{theorem}{\begin{tBox}\begin{theoremeT}}{\end{theoremeT}\end{tBox}}
\newenvironment{exercise}{\begin{eBox}\begin{exerciseT}}{\hfill{\color{ocre}\tiny\ensuremath{\blacksquare}}\end{exerciseT}\end{eBox}}				  
\newenvironment{definition}{\begin{dBox}\begin{definitionT}}{\end{definitionT}\end{dBox}}	
\newenvironment{example}{\begin{exampleT}}{\hfill{\tiny\ensuremath{\blacksquare}}\end{exampleT}}		
\newenvironment{corollary}{\begin{cBox}\begin{corollaryT}}{\end{corollaryT}\end{cBox}}	

%----------------------------------------------------------------------------------------
%	REMARK ENVIRONMENT
%----------------------------------------------------------------------------------------

\newenvironment{remark}{\par\vspace{10pt}\small % Vertical white space above the remark and smaller font size
\begin{list}{}{
\leftmargin=35pt % Indentation on the left
\rightmargin=25pt}\item\ignorespaces % Indentation on the right
\makebox[-2.5pt]{\begin{tikzpicture}[overlay]
\node[draw=ocre!60,line width=1pt,circle,fill=ocre!25,font=\sffamily\bfseries,inner sep=2pt,outer sep=0pt] at (-15pt,0pt){\textcolor{ocre}{R}};\end{tikzpicture}} % Orange R in a circle
\advance\baselineskip -1pt}{\end{list}\vskip5pt} % Tighter line spacing and white space after remark

%----------------------------------------------------------------------------------------
%	SECTION NUMBERING IN THE MARGIN
%----------------------------------------------------------------------------------------

\makeatletter
\renewcommand{\@seccntformat}[1]{\llap{\textcolor{ocre}{\csname the#1\endcsname}\hspace{1em}}}                    
\renewcommand{\section}{\@startsection{section}{1}{\z@}
{-4ex \@plus -1ex \@minus -.4ex}
{1ex \@plus.2ex }
{\normalfont\large\sffamily\bfseries}}
\renewcommand{\subsection}{\@startsection {subsection}{2}{\z@}
{-3ex \@plus -0.1ex \@minus -.4ex}
{0.5ex \@plus.2ex }
{\normalfont\sffamily\bfseries}}
\renewcommand{\subsubsection}{\@startsection {subsubsection}{3}{\z@}
{-2ex \@plus -0.1ex \@minus -.2ex}
{.2ex \@plus.2ex }
{\normalfont\small\sffamily\bfseries}}                        
\renewcommand\paragraph{\@startsection{paragraph}{4}{\z@}
{-2ex \@plus-.2ex \@minus .2ex}
{.1ex}
{\normalfont\small\sffamily\bfseries}}

%----------------------------------------------------------------------------------------
%	PART HEADINGS
%----------------------------------------------------------------------------------------

% Numbered part in the table of contents
\newcommand{\@mypartnumtocformat}[2]{%
	\setlength\fboxsep{0pt}%
	\noindent\colorbox{ocre!20}{\strut\parbox[c][.7cm]{\ecart}{\color{ocre!70}\Large\sffamily\bfseries\centering#1}}\hskip\esp\colorbox{ocre!40}{\strut\parbox[c][.7cm]{\linewidth-\ecart-\esp}{\Large\sffamily\centering#2}}%
}

% Unnumbered part in the table of contents
\newcommand{\@myparttocformat}[1]{%
	\setlength\fboxsep{0pt}%
	\noindent\colorbox{ocre!40}{\strut\parbox[c][.7cm]{\linewidth}{\Large\sffamily\centering#1}}%
}

\newlength\esp
\setlength\esp{4pt}
\newlength\ecart
\setlength\ecart{1.2cm-\esp}
\newcommand{\thepartimage}{}%
\newcommand{\partimage}[1]{\renewcommand{\thepartimage}{#1}}%
\def\@part[#1]#2{%
\ifnum \c@secnumdepth >-2\relax%
\refstepcounter{part}%
\addcontentsline{toc}{part}{\texorpdfstring{\protect\@mypartnumtocformat{\thepart}{#1}}{\partname~\thepart\ ---\ #1}}
\else%
\addcontentsline{toc}{part}{\texorpdfstring{\protect\@myparttocformat{#1}}{#1}}%
\fi%
\startcontents%
\markboth{}{}%
{\thispagestyle{empty}%
\begin{tikzpicture}[remember picture,overlay]%
\node at (current page.north west){\begin{tikzpicture}[remember picture,overlay]%	
\fill[ocre!20](0cm,0cm) rectangle (\paperwidth,-\paperheight);
\node[anchor=north] at (4cm,-3.25cm){\color{ocre!40}\fontsize{220}{100}\sffamily\bfseries\thepart}; 
\node[anchor=south east] at (\paperwidth-1cm,-\paperheight+1cm){\parbox[t][][t]{8.5cm}{
\printcontents{l}{0}{\setcounter{tocdepth}{1}}% The depth to which the Part mini table of contents displays headings; 0 for chapters only, 1 for chapters and sections and 2 for chapters, sections and subsections
}};
\node[anchor=north east] at (\paperwidth-1.5cm,-3.25cm){\parbox[t][][t]{15cm}{\strut\raggedleft\color{white}\fontsize{30}{30}\sffamily\bfseries#2}};
\end{tikzpicture}};
\end{tikzpicture}}%
\@endpart}
\def\@spart#1{%
\startcontents%
\phantomsection
{\thispagestyle{empty}%
\begin{tikzpicture}[remember picture,overlay]%
\node at (current page.north west){\begin{tikzpicture}[remember picture,overlay]%	
\fill[ocre!20](0cm,0cm) rectangle (\paperwidth,-\paperheight);
\node[anchor=north east] at (\paperwidth-1.5cm,-3.25cm){\parbox[t][][t]{15cm}{\strut\raggedleft\color{white}\fontsize{30}{30}\sffamily\bfseries#1}};
\end{tikzpicture}};
\end{tikzpicture}}
\addcontentsline{toc}{part}{\texorpdfstring{%
\setlength\fboxsep{0pt}%
\noindent\protect\colorbox{ocre!40}{\strut\protect\parbox[c][.7cm]{\linewidth}{\Large\sffamily\protect\centering #1\quad\mbox{}}}}{#1}}%
\@endpart}
\def\@endpart{\vfil\newpage
\if@twoside
\if@openright
\null
\thispagestyle{empty}%
\newpage
\fi
\fi
\if@tempswa
\twocolumn
\fi}

%----------------------------------------------------------------------------------------
%	CHAPTER HEADINGS
%----------------------------------------------------------------------------------------

% A switch to conditionally include a picture, implemented by Christian Hupfer
\newif\ifusechapterimage
\usechapterimagetrue
\newcommand{\thechapterimage}{}%
\newcommand{\chapterimage}[1]{\ifusechapterimage\renewcommand{\thechapterimage}{#1}\fi}%
\newcommand{\autodot}{.}
\def\@makechapterhead#1{%
{\parindent \z@ \raggedright \normalfont
\ifnum \c@secnumdepth >\m@ne
\if@mainmatter
\begin{tikzpicture}[remember picture,overlay]
\node at (current page.north west)
{\begin{tikzpicture}[remember picture,overlay]
\node[anchor=north west,inner sep=0pt] at (0,0) {\ifusechapterimage\includegraphics[width=\paperwidth]{\thechapterimage}\fi};
\draw[anchor=west] (\Gm@lmargin,-9cm) node [line width=2pt,rounded corners=15pt,draw=ocre,fill=white,fill opacity=0.5,inner sep=15pt]{\strut\makebox[22cm]{}};
\draw[anchor=west] (\Gm@lmargin+.3cm,-9cm) node {\huge\sffamily\bfseries\color{black}\thechapter\autodot~#1\strut};
\end{tikzpicture}};
\end{tikzpicture}
\else
\begin{tikzpicture}[remember picture,overlay]
\node at (current page.north west)
{\begin{tikzpicture}[remember picture,overlay]
\node[anchor=north west,inner sep=0pt] at (0,0) {\ifusechapterimage\includegraphics[width=\paperwidth]{\thechapterimage}\fi};
\draw[anchor=west] (\Gm@lmargin,-9cm) node [line width=2pt,rounded corners=15pt,draw=ocre,fill=white,fill opacity=0.5,inner sep=15pt]{\strut\makebox[22cm]{}};
\draw[anchor=west] (\Gm@lmargin+.3cm,-9cm) node {\huge\sffamily\bfseries\color{black}#1\strut};
\end{tikzpicture}};
\end{tikzpicture}
\fi\fi\par\vspace*{270\p@}}}

%-------------------------------------------

\def\@makeschapterhead#1{%
\begin{tikzpicture}[remember picture,overlay]
\node at (current page.north west)
{\begin{tikzpicture}[remember picture,overlay]
\node[anchor=north west,inner sep=0pt] at (0,0) {\ifusechapterimage\includegraphics[width=\paperwidth]{\thechapterimage}\fi};
\draw[anchor=west] (\Gm@lmargin,-9cm) node [line width=2pt,rounded corners=15pt,draw=ocre,fill=white,fill opacity=0.5,inner sep=15pt]{\strut\makebox[22cm]{}};
\draw[anchor=west] (\Gm@lmargin+.3cm,-9cm) node {\huge\sffamily\bfseries\color{black}#1\strut};
\end{tikzpicture}};
\end{tikzpicture}
\par\vspace*{270\p@}}
\makeatother

%----------------------------------------------------------------------------------------
%	LINKS
%----------------------------------------------------------------------------------------

\usepackage{verbatim}

\usepackage{hyperref}
\hypersetup{hidelinks,backref=true,pagebackref=true,hyperindex=true,colorlinks=false,breaklinks=true,urlcolor=ocre,bookmarks=true,bookmarksopen=false}

\usepackage{bookmark}
\bookmarksetup{
open,
numbered,
addtohook={%
\ifnum\bookmarkget{level}=0 % chapter
\bookmarksetup{bold}%
\fi
\ifnum\bookmarkget{level}=-1 % part
\bookmarksetup{color=ocre,bold}%
\fi
}
}
\begin{comment}
\begin{center}
\begin{multicols}\\
If x is \textbf{positive}:\\
\large{$\frac{(5)+1}{(5)-6}\geq\frac{(5)+2}{(5)-4}$}\\
\large{$\frac{6}{-1}\geq\frac{7}{1}$}\\
\large{$-6\geq7$}\\
\columnbreak
\normalsize{If x is \textbf{negative}:}\\
\large{$\frac{5}{5(-2)-5}\leq\frac{2}{(-2)-1}$}\\
\large{$\frac{5}{-30}\leq\frac{2}{-6}$}\\
\large{$\frac{-1}{6}\leq\frac{-1}{3}$}\\
\end{multicols}
\end{center}
\end{comment}

 % Insert the commands.tex file which contains the majority of the structure behind the template

\usepackage{multicol, wrapfig}
\graphicspath{{./Pictures/}}


\hypersetup{pdftitle={ApplicationofDerivativesPartTwo}pdfauthor={Joshua Bautista}}

%----------------------------------------------------------------------------------------

\begin{document}

%----------------------------------------------------------------------------------------
%   TITLE PAGE
%----------------------------------------------------------------------------------------
\part{Application of Derivatives Part Two\\ by: Joshua Bautista}
%----------------------------------------------------------------------------------------
%	CHAPTER 1
%----------------------------------------------------------------------------------------

\chapterimage{placeholder.JPG} % Chapter heading image

\pagebreak

\chapter{Application of Derivatives Part Two}

\section{Optimization}\index{Optimization}

For optimization problems...

\vspace*{5mm}

\noindent {\large \textbf{Example 1: Area}}

\noindent \small Sarah has a new puppy and she wants to maximize her outdoor time, so she builds her a fenced-in play area. She has 40 feet of fencing,
and she wants to fence off a rectangular area next to her house. The house will be one side of the play area, so that side needs no fencing. In order
for the puppy to have adequate space, the area needs to be at least 5  feet long and 5 feet wide. What is the largest area she could have? Is there a
smallest area she could have if she wants to use all 40 feet of fencing?

\vspace*{2.5mm}

\begin{enumerate}
    \item {\normalsize \textbf{Find out what you are trying to optimize.}} \\
          {\small In this example, we are trying to maximize and minimize the area enclosed.}
    \item {\normalsize\textbf{Draw a sketch.}}
          \begin{center}
              \includegraphics[scale=0.25]{Example1Pic.JPG}
          \end{center}
    \item {\normalsize\textbf{Write what you are trying to optimize as a function of the unknown variables idenitied in the picture.}}
          \begin{center}
              $A=lw$ \\
          \end{center}
          {\small We cannot take the derivative of this function and find the max/min as we must first get the function in terms of only one variable. Let's go back to the question and see
          what constraints are given. We are told that she has 40 feet of fencing, let's make an equation out of that.}
          \begin{center}
              $2w+l=40$ \\
              $w=\frac{40-l}{2}$ \\
          \end{center}
          {\small We have isolated $w$, now we can plug it back into the original function we made to make it in terms of only one variable.}
          \begin{center}
              $A=lw=(l)(\frac{40-l}{2})$ \\
          \end{center}
    \item {\normalsize\textbf{Find the critical points.}}
          \begin{center}
              $A(l)=10l-\frac{1}{2}l^2$ \\
              $A'(l)=10-l$ \\
              $20-l=0$ \\
              $l=20$ \\
          \end{center}
          {\small We must make sure that this critical point is not an extraneous solution. We need to make sure this value for $l$ lies in the domain of $l$. We are told
          the area must be at least 5 feet long and 5 feet wide. Plugging in $w=5$ in $2w+l=40$ gives us $l=30$}
          \begin{center}
              domain of $l: [5, 30]$
          \end{center}
          {\small $l=20$ is within the domain. Now, we must see if this is a local max, local min, or neither. We can check max with the 2nd derivative test.} \\
    \item {\normalsize\textbf{2nd derivative test.}}
          \begin{center}
              $A''(l)=-1$ \\
              $A''(20)=-1$
          \end{center}
          {\small Since the function is concave down (negative 2nd derivative) at any point $l$, $l=20$ must be a local maximum.} \\
    \item {\normalsize\textbf{Find minimum using EVT (plug-in endpoints of domain).}}
          \begin{multicols}{2}
              \begin{center}
                  At $l=5$: \\
                  $A(5) = 20(5)-\frac{1}{2}(5)^2$ \\
                  \textbf{$= 87.5 ft^2$}

                  \columnbreak

                  At $l=30$:
                  $A(30) = 20(30)-\frac{1}{2}(30)^2$ \\
                  $= 600-\frac{900}{2}$ \\
                  $=150 ft^2$

              \end{center}
          \end{multicols}

          \begin{multicols}{2}
              \begin{center}
                  $2w+5=40$ \\
                  $2w=35$ \\
                  $w=17.5$ \\

                  \columnbreak
                  $2w+20=40$ \\
                  $2w=20$ \\
                  $w=10$

              \end{center}
          \end{multicols}

\end{enumerate}
{\small $\therefore$ Maximum enclosed area at $l=20$ feet long and $w=10$ feet wide. Minimum enclosed area at $l=5$ feet long and $w=17.5$ feet wide.}

\pagebreak

\noindent {\large \textbf{Example 2: Distance}}

\noindent {\small A fence 8ft tall runs parellel to a tall building at a distance of 4ft from the building. What is the length of the shortest ladder
    that will reach from the ground over the fence to the wall of the building?}

\begin{center}
    \includegraphics[scale=0.4]{Example2Pic.JPG}
\end{center}

\noindent Once we draw/sketch the problem out, we must create an equation that will relate all of the variables. Note that we are minimizing $z$ (shortest ladder).
We have a right angled triangle, so we can use the \textbf{pythagorean theorem}.

\begin{center}
    $z^2=y^2+(4+x)^2$
\end{center}

\noindent This equation does not help as of right now as it contains two variables, we need it in terms of one variable. Let's use what else we are given - the smaller triangle.

\begin{center}
    \includegraphics[scale=0.5]{Example2Pic2.JPG}
\end{center}

\noindent These two triangles are \textbf{similar}, so we can use proportions to help us solve for y in terms of x.

\begin{center}
    \large{$\frac{y}{8}=\frac{4+x}{x}$ \\
    $yx = 8(4+x)$ \\
    $y=[\frac{8(4+x)}{x}]$}
\end{center}

\noindent Now, we can use this equation in the original equation we found at the start to get it in terms of one variable. It is very important to note
that if \textbf{z is minimized, $z^2$ is also minimized}. This is important because instead of isolating for $z^2$ and getting a square root on the other
side (which will be annoying to take the derivative of), we can simply take the derivative of $z^2$ and find the critical points.

\begin{center}
    $z^2=(\frac{8(4+x)}{x})^2+(4+x)^2$ \\
    $(z^2)'=2(\frac{8(4+x)}{x})(\frac{-32}{x^2})+2(4+x)$ \\
    $0=2(\frac{8(4+x)}{x})(\frac{-32}{x^2})+2(4+x)$ \\
    $-2(4+x)=2(\frac{8(4+x)}{x})(\frac{-32}{x^2})$ \\
    $2=2(\frac{8(4+x)}{x})(\frac{32}{x^2})$ \\
    $4+x=(\frac{8(4+x)}{x})(\frac{32}{x^2})$ \\
    $1=(\frac{8}{x})(\frac{32}{x^2})$ \\
    $1=(\frac{256}{x^3})$ \\
    $x=\sqrt[3]{256}$
\end{center}

\noindent \textbf{DOMAIN CHECK TODO}We need to make sure that this critical point is actually the minimum value. Let's do the 1st derivative test - plug in two numbers
into the derivative: one lower (4) and one higher (8) than $\sqrt[3]{256}$.

\begin{multicols}{2}
    \begin{center}
        \underline{$x=4$} \\
        $=2(\frac{8(6)}{4})(\frac{-32}{16})+2(8)$ \\
        $=$ negative

        \columnbreak
        \underline{$x=8$}\\
        $=2(\frac{8(12)}{8})(\frac{-32}{64})+2(12)$ \\
        $=$ positive
    \end{center}
\end{multicols}

\noindent By drawing what we found from the 1st derivative test, we can assume that at $x=\sqrt[3]{256}$, it is a minimum. Now let's find out the $y$ value.

\begin{center}
    $z^2=(\frac{8((\sqrt[3]{256})+4)}{\sqrt[3]{256}})^2+((\sqrt[3]{256})+4)^2$ \\
    $z^2=277.15$ \\
    $z=\sqrt{277.15}$ \\
    $z=16.65$
\end{center}

\noindent $\therefore$ The length of the shortest ladder that will reach will be $16.65$ft long.

\vspace*{5mm}

\noindent {\large \textbf{Example 3: Revenue}}

\noindent\small{The manager of a 100-unit apartment complex knows from experience that all units will be occupied if the rent is \$800 per month. A market survey suggests
    that, on average, one additional unit will remain vacant for each \$10 increase in rent. What rent should the manager charge to maximize revenue?}

\begin{center}
    Revenue = units sold x price per unit \\
    $R(x) = x \cdot p(x)$
\end{center}

\noindent We will need to find an expression for $p(x)$. Note that for every \$10 increase in price, there will be a 1 unit decrease. This is a contant/linear relationship: $-10$ slope.

\begin{center}
    $p(x) = -10x + b$
\end{center}

\noindent We are also given that the price of the units will be \$800 when all 100 units are being used.

\begin{center}
    $p(100)=800$
\end{center}

\noindent Bringing these two equations together we get:

\begin{center}
    $800=-10(100)+b$ \\
    $800=-1000+b$ \\
    $b=1800$ \\
    $\therefore p(x)=-10x+1800$
\end{center}

\noindent We can then use $p(x)$ in our original equation:

\begin{center}
    $R(x) = x \cdot p(x)$ \\
    $R(x)=x \cdot (-10x+1800)$ \\
    $R(x)=-10x^2+1800x$ \\
    $R'(x)=-20x+1800$
\end{center}

\noindent We can, as always, find the critical points by finding the x-values that make this equal to zero.

\begin{center}
    $0 =-20x+1800$ \\
    $-1800=-20x$ \\
    $x=90$ \\
\end{center}

\noindent To verify that this is the maximum, we need to plug in the endpoints of the domain as well (EVT): $0\leq x \leq 100$.

\begin{multicols}{3}
    \begin{center}
        $R(0)=-10(0)^2+1800(0)$ \\
        $R(0)=\$0$
        \columnbreak

        $R(90)=-10(90)^2+1800(90)$ \\
        $R(90)=-81000+162000$ \\
        $R(90)=\$81000$
        \columnbreak

        $R(100)=-10(100)^2+1800(100)$ \\
        $R(100)=-100000+180000$ \\
        $R(100)=\$80000$
    \end{center}
\end{multicols}

\noindent The maximum revenue will be \$81000 when 90 units are sold, so to find price per unit we need to divide revenue by number of units.

\begin{center}
    Revenue = units sold $\cdot$ price per unit \\
    Price per unit = revenue/units sold \\
    $= \frac{81000}{90}$ \\
    $=900$
\end{center}

\noindent $\therefore$ The manager should charge a \textbf{\$900} rent per month to maximize revenue to a value of \$81000.

\section{Related Rates}\index{Related Rates}

For related rates...

\vspace*{5mm}

\noindent {\large \textbf{Example 1: Area of Circle}}

\noindent When a circular metal plate is heated in an over, its radius increases at the rate of 0.02 cm/min. At what rate is the plate's area increasing
when the radius is 60cm?

\vspace*{2mm}

\noindent As always, let's first draw/sketch it out so we can have an easier time understanding the problem and what it is asking for: what rate the plate's area
is increasing \textbf{at the instant} (A.T.I) the radius is 60cm.

\begin{center}
    \includegraphics[scale=0.4]{Example3Pic.jpg}
\end{center}

\noindent Notice in the question it says that the radius \textbf{increases at a rate of 0.02 cm/min}. This basically gives us the derivative; whenever you see
'rate', it refers to the derivative. We are also given the radius.

\begin{center}
    \underline{given:}
    \begin{multicols}{2}
        $\frac{dr}{dt}=0.02 \frac{cm}{min}$

        $r=60cm$
    \end{multicols}
\end{center}

\noindent The question asks us \textbf{what rate is the plate's area increasing} when the radius is 60cm? Again, it is asking for the 'rate', so they are asking
for the derivative:

\begin{center}
    \underline{want:} \\
    \vspace*{1mm}
    $\frac{dA}{dt}=?$ \\
    \vspace*{1mm}
    A.T.I when r = 60 cm
\end{center}

\noindent Next, we need to find an equation that relates the area and the radius. Since we are working with circles, we already know this formula.

\begin{center}
    $A=\pi r^2$
\end{center}

\noindent Since we want $\frac{dA}{dt}$, we can simply take the derivative of this formula with respect to $t$ and after, plug in what we are given. \textbf{NOTE:} in the last line, remember that
we are taking the derivative with respect to $t$, so we cannot simply use the power rule.

\begin{center}
    $\frac{d}{dt}(A=\pi r^2)$ \\
    \vspace*{1mm}
    $\frac{dA}{dt}=2 \pi r \frac{dr}{dt}$ \\
    \vspace*{1mm}
    $\frac{dA}{dt}=2 \pi (60cm)(0.02\frac{cm}{min})$ \\
    \vspace*{1mm}
    $\frac{dA}{dt}=2.4 \pi \frac{cm^2}{min}$ \\
\end{center}

\noindent $\therefore$ At the instant when the radius is 60cm, the rate of which the plate's area is increasing by is $2.4 \pi \frac{cm^2}{min}$.

\noindent {\large \textbf{Example 2: Pythagorean}}

\noindent\small{Car A is travelling west at 50 mi/h and Car B is travelling north at 60 mi/h. Both are headed for the intersection of the two roads. At what rate are
    the cars approaching each other when Car A is 0.3 mi and Car B is 0.4 mi from the intersection?}

\vspace*{2mm}

\noindent Let's start by drawing a diagram to have a better understanding of what's happening. We will indicate the parts that are changing with red arrows and
label all the information given at the instant (A.T.I) the question is asking for.

\begin{center}
    \includegraphics[scale=0.5]{Example4Pic.jpg}
\end{center}

\noindent Reading the question again it tells us the rate of which the cars are travelling with respect to time. \textbf{NOTE:} the length of A and B are actually
decreasing, so the derivative should be negative as the distance is getting shorter - easy to see from the diagram.

\pagebreak

\begin{center}
    \underline{given:}
    \begin{multicols}{4}
        $\frac{dA}{dt}=-50 \frac{mi}{h}$

        $\frac{dB}{dt}=-60 \frac{mi}{h}$

        $A=0.3mi$

        $B=0.4mi$
    \end{multicols}
\end{center}

\noindent The question also asks, 'at what rate are the cars approaching each other?':

\begin{center}
    \underline{want:} \\
    $\frac{dc}{dt}=?$
\end{center}

\noindent Now, we need an equation that relates A, B, and C. Since we are working with triangles, it should be easy to notice that we can use \text{pythagorean's theorem}.
Everything that we are given and need ($\frac{dA}{dt}$, $\frac{dB}{dt}$, etc) are the derivatives of these variables with respect to time. So let's take the derivative of both sides.

\begin{center}
    $A^2+B^2=C^2$ \\
    \vspace*{1mm}
    $\frac{d}{dt}(A^2+B^2=C^2)$ \\
    \vspace*{1mm}
    $2A\frac{dA}{dt}+2B\frac{dB}{dt}=2C\frac{dC}{dt}$
\end{center}

\noindent To reiterate, it is very important to double-check if the rates are positive or negative. If Car A was going in the opposite direction, the rate would be
positive as the length/distance of A would be increasing. In this case, both Car A and Car B are going towards the intersection; the length/distance of A and B are
getting shorter, hence why $\frac{dA}{dt}$ and $\frac{dB}{dt}$ are negative. Let's plug in what we know now.

\begin{center}
    $2(0.3mi)(-50\frac{mi}{h})+2(0.4\frac{mi}{h})(-60mi)=2C\frac{dC}{dt}$ \\
\end{center}

\noindent You should notice that we do not have the value of C yet. Here we can use the equation we got from pythagorean's theorem.

\begin{center}
    $A^2+B^2=C^2$ \\
    \vspace*{1mm}
    $(0.3mi)^2+(0.4mi)^2=C^2$ \\
    \vspace*{1mm}
    $0.25mi^2=C^2$ \\
    \vspace*{1mm}
    $C=0.5mi$
\end{center}

\noindent We can continue where we left off and plug in this C value.

\begin{center}
    $2(0.3mi)(-50\frac{mi}{h})+2(0.4mi)(-60\frac{mi}{h})=2(0.5mi)\frac{dC}{dt}$ \\
    \vspace*{1mm}
    $(0.6mi)(-50\frac{mi}{h})+(0.8mi)(-60\frac{mi}{h})=(1mi)\frac{dC}{dt}$ \\
    \vspace*{1mm}
    $-78\frac{mi^2}{h}=(mi)\frac{dC}{dt}$ \\
    \vspace*{1mm}
    $-78\frac{mi}{h}=\frac{dC}{dt}$
\end{center}

\noindent $\therefore$ The cars the approaching each other at a rate of 78 $\frac{mi}{h}$.

\pagebreak

\noindent {\large \textbf{Example 3: Volume of Cone}}

\noindent {\small A water tank has the shape of an inverted (meaning up-side down) circular cone with base radius 2m and height 4m.
    If water is being pumped into the tank  at a rate of $2m^3/min$, find the rate at which the water level is rising when the water is 3m deep.}

\begin{center}
    \includegraphics[scale=0.45]{Example5Pic.jpg}
\end{center}

\noindent In the diagram, whatever is changing as time passes by is indicated in red (the height - $h$ and radius - $r$). We are given a rate: 'water is being pumped into the tank at a rate of 2
$\frac{m^3}{min}$. This is our $\frac{dV}{dt}$ as it is the rate of which the volume is changing as time passes by. We want to find 'the rate at which the water level is rising', which is
$\frac{dh}{dt}$; the rate at which the height is changing as time passes by.

\begin{multicols}{2}
    \begin{center}
        \underline{given:} \\
        $\frac{dV}{dt}=2\frac{m^3}{min}$

        \columnbreak
        \underline{want:} \\
        \vspace*{1mm}
        $\frac{dA}{dt}=?$ \\
        \vspace*{1mm}
        {\small A.T.I when h = 3m}
    \end{center}
\end{multicols}

\noindent Like the previous examples on related rates, we need to find an equation that can relate all of the variables together (V, r, h). We can use the volume of a cone formula: $V=\frac{1}{3} \pi r^2h$.
Like before, we need this in terms of only one variable. In the diagram there are two \emph{similar} triangles that can be used to set up a \textbf{proportion} - this is why it is important to draw a diagram.

\begin{center}
    \includegraphics[scale=0.4]{Example5Pic2.jpg} \\
    \vspace*{2mm}
    $\frac{2}{4}=\frac{r}{h}$ \\
    \vspace*{1mm}
    $\frac{1}{2}=\frac{r}{h}$ \\
    \vspace*{1mm}
    $r=\frac{1}{2}h$
\end{center}

\pagebreak

\noindent We can use this in the formula to get it in terms of one variable. After that, we can take the derivative of both sides, plug in what we are given, and solve for $\frac{dh}{dt}$.

\begin{center}
    $V_{cone}=\frac{1}{3} \pi r^2 h$ \\
    \vspace*{2mm}
    $V=\frac{1}{3} \pi (\frac{1}{2}h)^2 h$ \\
    \vspace*{2mm}
    $V=\frac{1}{3} \pi \frac{1}{4}h^2h$\\
    \vspace*{2mm}
    $\frac{d}{dt}(V=\frac{1}{12} \pi h^3)$ \\
    \vspace*{2mm}
    $\frac{dV}{dt}=\frac{3}{12} \pi h^2 \frac{dh}{dt}$ \\
    \vspace*{2mm}
    $2=\frac{1}{4} \pi (3)^2 \frac{dh}{dt}$ \\
    \vspace*{2mm}
    $2=\frac{9 \pi}{4} \frac{dh}{dt}$ \\
    \vspace*{2mm}
    $\frac{dh}{dt}=\frac{8}{9 \pi} \frac{m}{min}$
\end{center}

\noindent $\therefore$ The water level is rising at a rate of $\frac{8}{9 \pi} \frac{m}{min}$ when the water is 3m deep.

\section{L'Hopital's Rule (AP)}\index{Related Rates}

\includegraphics[scale=0.5]{LHopital.JPG}

\noindent L'Hopital's rule provides us a technique to \textbf{calculate the limit of indeterminate forms.} ($\frac{\infty}{\infty}$, $\frac{0}{0}$) When you have an indeterminate form, you can simply take the limit
of the derivative of the numerator divided by the derivative of the denominator. We can use L'Hopital's rule on the following forms:

\begin{multicols}{2}
    \begin{center}
        {\large$\frac{\infty}{\infty}$

            $\frac{0}{0}$}
    \end{center}
\end{multicols}

\noindent If you have other indeterminate forms, you can manipulate it to a form where you \emph{can} apply L'Hopital's rule.



\section{Approximations (AP)}\index{Related Rates}


\end{document}
