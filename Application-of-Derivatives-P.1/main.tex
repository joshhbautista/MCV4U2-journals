%%%%%%%%%%%%%%%%%%%%%%%%%%%%%%%%%%%%%%%%%
% The Legrand Orange Book
% LaTeX Template
% Version 2.4 (26/09/2018)
%
% This template was downloaded from:
% http://www.LaTeXTemplates.com
%
% Original author:
% Mathias Legrand (legrand.mathias@gmail.com) with modifications by:
% Vel (vel@latextemplates.com)
%
% License:
% CC BY-NC-SA 3.0 (http://creativecommons.org/licenses/by-nc-sa/3.0/)
%
% Compiling this template:
% This template uses biber for its bibliography and makeindex for its index.
% When you first open the template, compile it from the command line with the 
% commands below to make sure your LaTeX distribution is configured correctly:
%
% 1) pdflatex main
% 2) makeindex main.idx -s StyleInd.ist
% 3) biber main
% 4) pdflatex main x 2
%
% After this, when you wish to update the bibliography/index use the appropriate
% command above and make sure to compile with pdflatex several times 
% afterwards to propagate your changes to the document.
%
% This template also uses a number of packages which may need to be
% updated to the newest versions for the template to compile. It is strongly
% recommended you update your LaTeX distribution if you have any
% compilation errors.
%
% Important note:
% Chapter heading images should have a 2:1 width:height ratio,
% e.g. 920px width and 460px height.
%
%%%%%%%%%%%%%%%%%%%%%%%%%%%%%%%%%%%%%%%%%

%----------------------------------------------------------------------------------------
%	PACKAGES AND OTHER DOCUMENT CONFIGURATIONS
%----------------------------------------------------------------------------------------

\documentclass[12pt,fleqn]{book} % Default font size and left-justified equations

\input{structure.tex} % Insert the commands.tex file which contains the majority of the structure behind the template

\usepackage{multicol, wrapfig}
\graphicspath{{./Pictures/}}


\hypersetup{pdftitle={ApplicationofDerivativesPartOne}pdfauthor={Joshua Bautista}}

%----------------------------------------------------------------------------------------

\begin{document}

%----------------------------------------------------------------------------------------
%   TITLE PAGE
%----------------------------------------------------------------------------------------
\part{Application of Derivatives Part One\\ by: Joshua Bautista}
%----------------------------------------------------------------------------------------
%	CHAPTER 1
%----------------------------------------------------------------------------------------

\chapterimage{VelAccelPic.JPG} % Chapter heading image

\chapter{Application of Derivatives Part One}

\vspace*{-20mm}

\section{Sketching}\index{Sketching}

It is important to sketch out the function in the question to reveal all of its qualities (increasing/decreasing intervals, concave up/down, inflection points, etc).
There is an algorithm to determine all of the details of the graph.

\vspace*{3mm}

\begin{enumerate}
    \item From the original graph:
          \begin{itemize}
              \item You must first \textbf{factor} to check if any \textbf{holes} are in the graph.
              \item State \textbf{VA's} and \textbf{Domain}.
              \item Find the \textbf{x and y intercepts}.
              \item Find the \textbf{end behaviour}.
              \item Look at the behaviour near \textbf{zeros} (x-intercepts) and \textbf{VA's}. (Remember - do this by looking at multiplicities of zeros)
              \item *CAN BE SKIPPED* - Find \textbf{positive and negative intervals} between zeros and VA's.
          \end{itemize}

          \vspace*{4mm}

    \item From the first derivative:
          \begin{itemize}
              \item Find \textbf{critical points}.
              \item *CAN BE SKIPPED* - Find \textbf{increasing/decreasing intervals}.
          \end{itemize}

          \vspace*{4mm}

    \item From the second derivative:
          \begin{itemize}
              \item Find possible \textbf{inflection points}.
              \item Find \textbf{concave up/down intervals}.
              \item Decide if the possible inflection points found are actual inflection points and classify the critical points using the 2nd derivative test.
          \end{itemize}
\end{enumerate}

\pagebreak

\noindent {\large \textbf{Example 1:}}

\vspace*{3mm}

\noindent Sketch and label all intercepts, asymtotes, crticial points and inflection points. Show all justifying steps.

\vspace*{3mm}

{\large $y=(x^\frac{1}{3})(x-4)$}

\vspace*{3mm}

\noindent \textbf{1.} Cannot be factored any further. \textbf{$\therefore$ no holes, or VA's. Domain $x \in \Re$} \\

\vspace*{-4mm}

\noindent \textbf{1.} Find x and y intercepts:

\vspace*{-2mm}

\begin{multicols}{2}
    \begin{center}
        \underline{x-intercept} \\
        $0=(x^\frac{1}{3})(x-4)$\\
        by property of zeros: \\
        \textbf{$\therefore (0, 0)$ and $(4, 0)$}

        \columnbreak

        \underline{y-intercept} \\
        $y=((0)^\frac{1}{3})((0)-4)$ \\
        $y=(0)(-4)$ \\
        $y=0$ \\
        \textbf{$\therefore (0, 0)$}
        \columnbreak
    \end{center}
\end{multicols}

\noindent \textbf{1.} Find end behaviour: The function is not rational/exponential. $\therefore$ no HA/OA.

\vspace*{2mm}

\noindent \textbf{2.} Find critical points: Find $y'$ (first derivative) and find when it is equal to 0 or DNE.

\vspace*{-2mm}

\begin{center}
    $y = (x^\frac{1}{3})(x-4)$ \\
    \vspace*{2mm}
    $y' = (x^\frac{1}{3})(1) + (x-4)(\frac{1}{3})(x^\frac{-2}{3})$ \\
    \vspace*{1mm}
    $= (x^\frac{-2}{3})[x+\frac{1}{3}(x-4)]$ \\
    \vspace*{1mm}
    $= (x^\frac{-2}{3})[x+\frac{(x-4)}{3}]$ \\
    \vspace*{1mm}
    $= (x^\frac{-2}{3})(\frac{3x+x-4}{3})$ \\
    \vspace*{1mm}
    {\large $= \frac{4(x-1)}{3x^\frac{2}{3}}$}
    \vspace*{1mm}
\end{center}

\begin{multicols}{2}
    \begin{center}
        $0= \frac{4(x-1)}{3x^\frac{2}{3}}$ \\
        \vspace*{1mm}
        $0 = 4(x-1)$ \\
        \vspace*{1mm}
        $\therefore$ critical pt. $x = 1$
        \columnbreak

        DNE when denominator = 0 \\
        \vspace*{1mm}
        $0 = 3x^\frac{2}{3}$ \\
        \vspace*{1mm}
        $\therefore$ critical pt. $x = 0$
    \end{center}
\end{multicols}



%----------------------------------------------------------------------------------------

\section{Velocity and Acceleration}\index{Notation}

\includegraphics[scale=0.4]{VelAccelPic.JPG}

\noindent Before we start doing problems involving velocity and acceleration, we must make sure we know these key definitions:

\vspace*{2mm}

\noindent{\large\textbf{Displacement}} \\
Displacement is the \textbf{change in position} of an object. It is concerned with the initial position of an object to its final position.

\begin{center}
    {\large $Displacement = s(t)$}

\end{center}

\vspace*{2mm}

\noindent{\large\textbf{Velocity}} \\
Velocity is speed over time.

\begin{center}
    {\large $Velocity = v(t) = \frac{ds}{dt} = \frac{\Delta s}{\Delta t}$}

\end{center}

\vspace*{2mm}

\noindent{\large\textbf{Acceleration}} \\
Acceleration is concerned

\begin{center}
    negative acceleration $\to$ decreasing velocity. \\
    position acceleration $\to$ increasing velocity.
\end{center}

\begin{center}
    {\large $Acceleration = a(t) = \frac{d^2v}{dt^2}$}

\end{center}

\vspace*{2mm}

\noindent{\large\textbf{Jerk/Turbulence:}} \\

\begin{center}
    {\large $Jerk/Turbulence = j(t) = \frac{da}{dt} = \frac{d^2v}{dt^2} = \frac{d^3s}{dt^3}$}
\end{center}


\vspace*{2mm}


%----------------------------------------------------------------------------------------

\section{Other Applications}\index{Operations with Inequalities}

\begin{center}

\end{center}

\vspace*{-3mm}

\pagebreak

\begin{center}

\end{center}

\end{document}
