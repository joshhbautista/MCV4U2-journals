%%%%%%%%%%%%%%%%%%%%%%%%%%%%%%%%%%%%%%%%%
% The Legrand Orange Book
% LaTeX Template
% Version 2.4 (26/09/2018)
%
% This template was downloaded from:
% http://www.LaTeXTemplates.com
%
% Original author:
% Mathias Legrand (legrand.mathias@gmail.com) with modifications by:
% Vel (vel@latextemplates.com)
%
% License:
% CC BY-NC-SA 3.0 (http://creativecommons.org/licenses/by-nc-sa/3.0/)
%
% Compiling this template:
% This template uses biber for its bibliography and makeindex for its index.
% When you first open the template, compile it from the command line with the 
% commands below to make sure your LaTeX distribution is configured correctly:
%
% 1) pdflatex main
% 2) makeindex main.idx -s StyleInd.ist
% 3) biber main
% 4) pdflatex main x 2
%
% After this, when you wish to update the bibliography/index use the appropriate
% command above and make sure to compile with pdflatex several times 
% afterwards to propagate your changes to the document.
%
% This template also uses a number of packages which may need to be
% updated to the newest versions for the template to compile. It is strongly
% recommended you update your LaTeX distribution if you have any
% compilation errors.
%
% Important note:
% Chapter heading images should have a 2:1 width:height ratio,
% e.g. 920px width and 460px height.
%
%%%%%%%%%%%%%%%%%%%%%%%%%%%%%%%%%%%%%%%%%

%----------------------------------------------------------------------------------------
%	PACKAGES AND OTHER DOCUMENT CONFIGURATIONS
%----------------------------------------------------------------------------------------

\documentclass[12pt,fleqn]{book} % Default font size and left-justified equations

%%%%%%%%%%%%%%%%%%%%%%%%%%%%%%%%%%%%%%%%%
% The Legrand Orange Book
% Structural Definitions File
% Version 2.1 (26/09/2018)
%
% Original author:
% Mathias Legrand (legrand.mathias@gmail.com) with modifications by:
% Vel (vel@latextemplates.com)
% 
% This file was downloaded from:
% http://www.LaTeXTemplates.com
%
% License:
% CC BY-NC-SA 3.0 (http://creativecommons.org/licenses/by-nc-sa/3.0/)
%
%%%%%%%%%%%%%%%%%%%%%%%%%%%%%%%%%%%%%%%%%

%----------------------------------------------------------------------------------------
%	VARIOUS REQUIRED PACKAGES AND CONFIGURATIONS
%----------------------------------------------------------------------------------------

\usepackage{graphicx} % Required for including pictures
\graphicspath{{Pictures/}} % Specifies the directory where pictures are stored

\usepackage{lipsum} % Inserts dummy text

\usepackage{tikz} % Required for drawing custom shapes

\usepackage[english]{babel} % English language/hyphenation

\usepackage{enumitem} % Customize lists
\setlist{nolistsep} % Reduce spacing between bullet points and numbered lists

\usepackage{booktabs} % Required for nicer horizontal rules in tables

\usepackage{xcolor} % Required for specifying colors by name
\definecolor{ocre}{RGB}{243,102,25} % Define the orange color used for highlighting throughout the book


%----------------------------------------------------------------------------------------
%	MARGINS
%----------------------------------------------------------------------------------------

\usepackage{geometry} % Required for adjusting page dimensions and margins

\geometry{
	paper=a4paper, % Paper size, change to letterpaper for US letter size
	top=3cm, % Top margin
	bottom=3cm, % Bottom margin
	left=3cm, % Left margin
	right=3cm, % Right margin
	headheight=14pt, % Header height
	footskip=1.4cm, % Space from the bottom margin to the baseline of the footer
	headsep=10pt, % Space from the top margin to the baseline of the header
	%showframe, % Uncomment to show how the type block is set on the page
}

%----------------------------------------------------------------------------------------
%	FONTS
%----------------------------------------------------------------------------------------

\usepackage{avant} % Use the Avantgarde font for headings
%\usepackage{times} % Use the Times font for headings
\usepackage{mathptmx} % Use the Adobe Times Roman as the default text font together with math symbols from the Sym­bol, Chancery and Com­puter Modern fonts

\usepackage{microtype} % Slightly tweak font spacing for aesthetics
\usepackage[utf8]{inputenc} % Required for including letters with accents
\usepackage[T1]{fontenc} % Use 8-bit encoding that has 256 glyphs

%----------------------------------------------------------------------------------------
%	BIBLIOGRAPHY AND INDEX
%----------------------------------------------------------------------------------------

\usepackage[style=numeric,citestyle=numeric,sorting=nyt,sortcites=true,autopunct=true,babel=hyphen,hyperref=true,abbreviate=false,backref=true,backend=biber]{biblatex}
\addbibresource{bibliography.bib} % BibTeX bibliography file
\defbibheading{bibempty}{}

\usepackage{calc} % For simpler calculation - used for spacing the index letter headings correctly
\usepackage{makeidx} % Required to make an index
\makeindex % Tells LaTeX to create the files required for indexing

%----------------------------------------------------------------------------------------
%	MAIN TABLE OF CONTENTS
%----------------------------------------------------------------------------------------

\usepackage{titletoc} % Required for manipulating the table of contents

\contentsmargin{0cm} % Removes the default margin

% Part text styling (this is mostly taken care of in the PART HEADINGS section of this file)
\titlecontents{part}
	[0cm] % Left indentation
	{\addvspace{20pt}\bfseries} % Spacing and font options for parts
	{}
	{}
	{}

% Chapter text styling
\titlecontents{chapter}
	[1.25cm] % Left indentation
	{\addvspace{12pt}\large\sffamily\bfseries} % Spacing and font options for chapters
	{\color{ocre!60}\contentslabel[\Large\thecontentslabel]{1.25cm}\color{ocre}} % Formatting of numbered sections of this type
	{\color{ocre}} % Formatting of numberless sections of this type
	{\color{ocre!60}\normalsize\;\titlerule*[.5pc]{.}\;\thecontentspage} % Formatting of the filler to the right of the heading and the page number

% Section text styling
\titlecontents{section}
	[1.25cm] % Left indentation
	{\addvspace{3pt}\sffamily\bfseries} % Spacing and font options for sections
	{\contentslabel[\thecontentslabel]{1.25cm}} % Formatting of numbered sections of this type
	{} % Formatting of numberless sections of this type
	{\hfill\color{black}\thecontentspage} % Formatting of the filler to the right of the heading and the page number

% Subsection text styling
\titlecontents{subsection}
	[1.25cm] % Left indentation
	{\addvspace{1pt}\sffamily\small} % Spacing and font options for subsections
	{\contentslabel[\thecontentslabel]{1.25cm}} % Formatting of numbered sections of this type
	{} % Formatting of numberless sections of this type
	{\ \titlerule*[.5pc]{.}\;\thecontentspage} % Formatting of the filler to the right of the heading and the page number

% Figure text styling
\titlecontents{figure}
	[1.25cm] % Left indentation
	{\addvspace{1pt}\sffamily\small} % Spacing and font options for figures
	{\thecontentslabel\hspace*{1em}} % Formatting of numbered sections of this type
	{} % Formatting of numberless sections of this type
	{\ \titlerule*[.5pc]{.}\;\thecontentspage} % Formatting of the filler to the right of the heading and the page number

% Table text styling
\titlecontents{table}
	[1.25cm] % Left indentation
	{\addvspace{1pt}\sffamily\small} % Spacing and font options for tables
	{\thecontentslabel\hspace*{1em}} % Formatting of numbered sections of this type
	{} % Formatting of numberless sections of this type
	{\ \titlerule*[.5pc]{.}\;\thecontentspage} % Formatting of the filler to the right of the heading and the page number

%----------------------------------------------------------------------------------------
%	MINI TABLE OF CONTENTS IN PART HEADS
%----------------------------------------------------------------------------------------

% Chapter text styling
\titlecontents{lchapter}
	[0em] % Left indentation
	{\addvspace{15pt}\large\sffamily\bfseries} % Spacing and font options for chapters
	{\color{ocre}\contentslabel[\Large\thecontentslabel]{1.25cm}\color{ocre}} % Chapter number
	{}  
	{\color{ocre}\normalsize\sffamily\bfseries\;\titlerule*[.5pc]{.}\;\thecontentspage} % Page number

% Section text styling
\titlecontents{lsection}
	[0em] % Left indentation
	{\sffamily\small} % Spacing and font options for sections
	{\contentslabel[\thecontentslabel]{1.25cm}} % Section number
	{}
	{}

% Subsection text styling (note these aren't shown by default, display them by searchings this file for tocdepth and reading the commented text)
\titlecontents{lsubsection}
	[.5em] % Left indentation
	{\sffamily\footnotesize} % Spacing and font options for subsections
	{\contentslabel[\thecontentslabel]{1.25cm}}
	{}
	{}

%----------------------------------------------------------------------------------------
%	HEADERS AND FOOTERS
%----------------------------------------------------------------------------------------

\usepackage{fancyhdr} % Required for header and footer configuration

\pagestyle{fancy} % Enable the custom headers and footers

\renewcommand{\chaptermark}[1]{\markboth{\sffamily\normalsize\bfseries\chaptername\ \thechapter.\ #1}{}} % Styling for the current chapter in the header
\renewcommand{\sectionmark}[1]{\markright{\sffamily\normalsize\thesection\hspace{5pt}#1}{}} % Styling for the current section in the header

\fancyhf{} % Clear default headers and footers
\fancyhead[LE,RO]{\sffamily\normalsize\thepage} % Styling for the page number in the header
\fancyhead[LO]{\rightmark} % Print the nearest section name on the left side of odd pages
\fancyhead[RE]{\leftmark} % Print the current chapter name on the right side of even pages
%\fancyfoot[C]{\thepage} % Uncomment to include a footer

\renewcommand{\headrulewidth}{0.5pt} % Thickness of the rule under the header

\fancypagestyle{plain}{% Style for when a plain pagestyle is specified
	\fancyhead{}\renewcommand{\headrulewidth}{0pt}%
}

% Removes the header from odd empty pages at the end of chapters
\makeatletter
\renewcommand{\cleardoublepage}{
\clearpage\ifodd\c@page\else
\hbox{}
\vspace*{\fill}
\thispagestyle{empty}
\newpage
\fi}

%----------------------------------------------------------------------------------------
%	THEOREM STYLES
%----------------------------------------------------------------------------------------

\usepackage{tikz}
\newcommand\pacman[2]{\tikz[baseline, #1]{%
    \draw[thick,fill=#2]
    (0,0) -- (30:1cm) arc (30:330:1cm) -- cycle;
    \fill (0,2/3) circle (1.5mm);}
                       } 
                      

\usepackage{amsmath,amsfonts,amssymb,amsthm} % For math equations, theorems, symbols, etc

\newcommand{\intoo}[2]{\mathopen{]}#1\,;#2\mathclose{[}}
\newcommand{\ud}{\mathop{\mathrm{{}d}}\mathopen{}}
\newcommand{\intff}[2]{\mathopen{[}#1\,;#2\mathclose{]}}
\renewcommand{\qedsymbol}{$\blacksquare$}
\newtheorem{notation}{Notation}[chapter]

% Boxed/framed environments
\newtheoremstyle{ocrenumbox}% Theorem style name
{0pt}% Space above
{0pt}% Space below
{\normalfont}% Body font
{}% Indent amount
{\small\bf\sffamily\color{ocre}}% Theorem head font
{\;}% Punctuation after theorem head
{0.25em}% Space after theorem head
{\small\sffamily\color{ocre}\thmname{#1}\nobreakspace\thmnumber{\@ifnotempty{#1}{}\@upn{#2}}% Theorem text (e.g. Theorem 2.1)
\thmnote{\nobreakspace\the\thm@notefont\sffamily\bfseries\color{black}---\nobreakspace#3.}} % Optional theorem note

\newtheoremstyle{blacknumex}% Theorem style name
{5pt}% Space above
{5pt}% Space below
{\normalfont}% Body font
{} % Indent amount
{\small\bf\sffamily}% Theorem head font
{\;}% Punctuation after theorem head
{0.25em}% Space after theorem head
{\small\sffamily{\tiny\ensuremath{\blacksquare}}\nobreakspace\thmname{#1}\nobreakspace\thmnumber{\@ifnotempty{#1}{}\@upn{#2}}% Theorem text (e.g. Theorem 2.1)
\thmnote{\nobreakspace\the\thm@notefont\sffamily\bfseries---\nobreakspace#3.}}% Optional theorem note

\newtheoremstyle{blacknumbox} % Theorem style name
{0pt}% Space above
{0pt}% Space below
{\normalfont}% Body font
{}% Indent amount
{\small\bf\sffamily}% Theorem head font
{\;}% Punctuation after theorem head
{0.25em}% Space after theorem head
{\small\sffamily\thmname{#1}\nobreakspace\thmnumber{\@ifnotempty{#1}{}\@upn{#2}}% Theorem text (e.g. Theorem 2.1)
\thmnote{\nobreakspace\the\thm@notefont\sffamily\bfseries---\nobreakspace#3.}}% Optional theorem note

% Non-boxed/non-framed environments
\newtheoremstyle{ocrenum}% Theorem style name
{5pt}% Space above
{5pt}% Space below
{\normalfont}% Body font
{}% Indent amount
{\small\bf\sffamily\color{ocre}}% Theorem head font
{\;}% Punctuation after theorem head
{0.25em}% Space after theorem head
{\small\sffamily\color{ocre}\thmname{#1}\nobreakspace\thmnumber{\@ifnotempty{#1}{}\@upn{#2}}% Theorem text (e.g. Theorem 2.1)
\thmnote{\nobreakspace\the\thm@notefont\sffamily\bfseries\color{black}---\nobreakspace#3.}} % Optional theorem note
\makeatother

% Defines the theorem text style for each type of theorem to one of the three styles above
\newcounter{dummy} 
\numberwithin{dummy}{section}
\theoremstyle{ocrenumbox}
\newtheorem{theoremeT}[dummy]{Theorem}
\newtheorem{problem}{Problem}[chapter]
\newtheorem{exerciseT}{Exercise}[chapter]
\theoremstyle{blacknumex}
\newtheorem{exampleT}{Example}[chapter]
\theoremstyle{blacknumbox}
\newtheorem{vocabulary}{Vocabulary}[chapter]
\newtheorem{definitionT}{Definition}[section]
\newtheorem{corollaryT}[dummy]{Corollary}
\theoremstyle{ocrenum}
\newtheorem{proposition}[dummy]{Proposition}

%----------------------------------------------------------------------------------------
%	DEFINITION OF COLORED BOXES
%----------------------------------------------------------------------------------------

\RequirePackage[framemethod=default]{mdframed} % Required for creating the theorem, definition, exercise and corollary boxes

% Theorem box
\newmdenv[skipabove=7pt,
skipbelow=7pt,
backgroundcolor=black!5,
linecolor=ocre,
innerleftmargin=5pt,
innerrightmargin=5pt,
innertopmargin=5pt,
leftmargin=0cm,
rightmargin=0cm,
innerbottommargin=5pt]{tBox}

% Exercise box	  
\newmdenv[skipabove=7pt,
skipbelow=7pt,
rightline=false,
leftline=true,
topline=false,
bottomline=false,
backgroundcolor=ocre!10,
linecolor=ocre,
innerleftmargin=5pt,
innerrightmargin=5pt,
innertopmargin=5pt,
innerbottommargin=5pt,
leftmargin=0cm,
rightmargin=0cm,
linewidth=4pt]{eBox}	

% Definition box
\newmdenv[skipabove=7pt,
skipbelow=7pt,
rightline=false,
leftline=true,
topline=false,
bottomline=false,
linecolor=ocre,
innerleftmargin=5pt,
innerrightmargin=5pt,
innertopmargin=0pt,
leftmargin=0cm,
rightmargin=0cm,
linewidth=4pt,
innerbottommargin=0pt]{dBox}	

% Corollary box
\newmdenv[skipabove=7pt,
skipbelow=7pt,
rightline=false,
leftline=true,
topline=false,
bottomline=false,
linecolor=gray,
backgroundcolor=black!5,
innerleftmargin=5pt,
innerrightmargin=5pt,
innertopmargin=5pt,
leftmargin=0cm,
rightmargin=0cm,
linewidth=4pt,
innerbottommargin=5pt]{cBox}

% Creates an environment for each type of theorem and assigns it a theorem text style from the "Theorem Styles" section above and a colored box from above
\newenvironment{theorem}{\begin{tBox}\begin{theoremeT}}{\end{theoremeT}\end{tBox}}
\newenvironment{exercise}{\begin{eBox}\begin{exerciseT}}{\hfill{\color{ocre}\tiny\ensuremath{\blacksquare}}\end{exerciseT}\end{eBox}}				  
\newenvironment{definition}{\begin{dBox}\begin{definitionT}}{\end{definitionT}\end{dBox}}	
\newenvironment{example}{\begin{exampleT}}{\hfill{\tiny\ensuremath{\blacksquare}}\end{exampleT}}		
\newenvironment{corollary}{\begin{cBox}\begin{corollaryT}}{\end{corollaryT}\end{cBox}}	

%----------------------------------------------------------------------------------------
%	REMARK ENVIRONMENT
%----------------------------------------------------------------------------------------

\newenvironment{remark}{\par\vspace{10pt}\small % Vertical white space above the remark and smaller font size
\begin{list}{}{
\leftmargin=35pt % Indentation on the left
\rightmargin=25pt}\item\ignorespaces % Indentation on the right
\makebox[-2.5pt]{\begin{tikzpicture}[overlay]
\node[draw=ocre!60,line width=1pt,circle,fill=ocre!25,font=\sffamily\bfseries,inner sep=2pt,outer sep=0pt] at (-15pt,0pt){\textcolor{ocre}{R}};\end{tikzpicture}} % Orange R in a circle
\advance\baselineskip -1pt}{\end{list}\vskip5pt} % Tighter line spacing and white space after remark

%----------------------------------------------------------------------------------------
%	SECTION NUMBERING IN THE MARGIN
%----------------------------------------------------------------------------------------

\makeatletter
\renewcommand{\@seccntformat}[1]{\llap{\textcolor{ocre}{\csname the#1\endcsname}\hspace{1em}}}                    
\renewcommand{\section}{\@startsection{section}{1}{\z@}
{-4ex \@plus -1ex \@minus -.4ex}
{1ex \@plus.2ex }
{\normalfont\large\sffamily\bfseries}}
\renewcommand{\subsection}{\@startsection {subsection}{2}{\z@}
{-3ex \@plus -0.1ex \@minus -.4ex}
{0.5ex \@plus.2ex }
{\normalfont\sffamily\bfseries}}
\renewcommand{\subsubsection}{\@startsection {subsubsection}{3}{\z@}
{-2ex \@plus -0.1ex \@minus -.2ex}
{.2ex \@plus.2ex }
{\normalfont\small\sffamily\bfseries}}                        
\renewcommand\paragraph{\@startsection{paragraph}{4}{\z@}
{-2ex \@plus-.2ex \@minus .2ex}
{.1ex}
{\normalfont\small\sffamily\bfseries}}

%----------------------------------------------------------------------------------------
%	PART HEADINGS
%----------------------------------------------------------------------------------------

% Numbered part in the table of contents
\newcommand{\@mypartnumtocformat}[2]{%
	\setlength\fboxsep{0pt}%
	\noindent\colorbox{ocre!20}{\strut\parbox[c][.7cm]{\ecart}{\color{ocre!70}\Large\sffamily\bfseries\centering#1}}\hskip\esp\colorbox{ocre!40}{\strut\parbox[c][.7cm]{\linewidth-\ecart-\esp}{\Large\sffamily\centering#2}}%
}

% Unnumbered part in the table of contents
\newcommand{\@myparttocformat}[1]{%
	\setlength\fboxsep{0pt}%
	\noindent\colorbox{ocre!40}{\strut\parbox[c][.7cm]{\linewidth}{\Large\sffamily\centering#1}}%
}

\newlength\esp
\setlength\esp{4pt}
\newlength\ecart
\setlength\ecart{1.2cm-\esp}
\newcommand{\thepartimage}{}%
\newcommand{\partimage}[1]{\renewcommand{\thepartimage}{#1}}%
\def\@part[#1]#2{%
\ifnum \c@secnumdepth >-2\relax%
\refstepcounter{part}%
\addcontentsline{toc}{part}{\texorpdfstring{\protect\@mypartnumtocformat{\thepart}{#1}}{\partname~\thepart\ ---\ #1}}
\else%
\addcontentsline{toc}{part}{\texorpdfstring{\protect\@myparttocformat{#1}}{#1}}%
\fi%
\startcontents%
\markboth{}{}%
{\thispagestyle{empty}%
\begin{tikzpicture}[remember picture,overlay]%
\node at (current page.north west){\begin{tikzpicture}[remember picture,overlay]%	
\fill[ocre!20](0cm,0cm) rectangle (\paperwidth,-\paperheight);
\node[anchor=north] at (4cm,-3.25cm){\color{ocre!40}\fontsize{220}{100}\sffamily\bfseries\thepart}; 
\node[anchor=south east] at (\paperwidth-1cm,-\paperheight+1cm){\parbox[t][][t]{8.5cm}{
\printcontents{l}{0}{\setcounter{tocdepth}{1}}% The depth to which the Part mini table of contents displays headings; 0 for chapters only, 1 for chapters and sections and 2 for chapters, sections and subsections
}};
\node[anchor=north east] at (\paperwidth-1.5cm,-3.25cm){\parbox[t][][t]{15cm}{\strut\raggedleft\color{white}\fontsize{30}{30}\sffamily\bfseries#2}};
\end{tikzpicture}};
\end{tikzpicture}}%
\@endpart}
\def\@spart#1{%
\startcontents%
\phantomsection
{\thispagestyle{empty}%
\begin{tikzpicture}[remember picture,overlay]%
\node at (current page.north west){\begin{tikzpicture}[remember picture,overlay]%	
\fill[ocre!20](0cm,0cm) rectangle (\paperwidth,-\paperheight);
\node[anchor=north east] at (\paperwidth-1.5cm,-3.25cm){\parbox[t][][t]{15cm}{\strut\raggedleft\color{white}\fontsize{30}{30}\sffamily\bfseries#1}};
\end{tikzpicture}};
\end{tikzpicture}}
\addcontentsline{toc}{part}{\texorpdfstring{%
\setlength\fboxsep{0pt}%
\noindent\protect\colorbox{ocre!40}{\strut\protect\parbox[c][.7cm]{\linewidth}{\Large\sffamily\protect\centering #1\quad\mbox{}}}}{#1}}%
\@endpart}
\def\@endpart{\vfil\newpage
\if@twoside
\if@openright
\null
\thispagestyle{empty}%
\newpage
\fi
\fi
\if@tempswa
\twocolumn
\fi}

%----------------------------------------------------------------------------------------
%	CHAPTER HEADINGS
%----------------------------------------------------------------------------------------

% A switch to conditionally include a picture, implemented by Christian Hupfer
\newif\ifusechapterimage
\usechapterimagetrue
\newcommand{\thechapterimage}{}%
\newcommand{\chapterimage}[1]{\ifusechapterimage\renewcommand{\thechapterimage}{#1}\fi}%
\newcommand{\autodot}{.}
\def\@makechapterhead#1{%
{\parindent \z@ \raggedright \normalfont
\ifnum \c@secnumdepth >\m@ne
\if@mainmatter
\begin{tikzpicture}[remember picture,overlay]
\node at (current page.north west)
{\begin{tikzpicture}[remember picture,overlay]
\node[anchor=north west,inner sep=0pt] at (0,0) {\ifusechapterimage\includegraphics[width=\paperwidth]{\thechapterimage}\fi};
\draw[anchor=west] (\Gm@lmargin,-9cm) node [line width=2pt,rounded corners=15pt,draw=ocre,fill=white,fill opacity=0.5,inner sep=15pt]{\strut\makebox[22cm]{}};
\draw[anchor=west] (\Gm@lmargin+.3cm,-9cm) node {\huge\sffamily\bfseries\color{black}\thechapter\autodot~#1\strut};
\end{tikzpicture}};
\end{tikzpicture}
\else
\begin{tikzpicture}[remember picture,overlay]
\node at (current page.north west)
{\begin{tikzpicture}[remember picture,overlay]
\node[anchor=north west,inner sep=0pt] at (0,0) {\ifusechapterimage\includegraphics[width=\paperwidth]{\thechapterimage}\fi};
\draw[anchor=west] (\Gm@lmargin,-9cm) node [line width=2pt,rounded corners=15pt,draw=ocre,fill=white,fill opacity=0.5,inner sep=15pt]{\strut\makebox[22cm]{}};
\draw[anchor=west] (\Gm@lmargin+.3cm,-9cm) node {\huge\sffamily\bfseries\color{black}#1\strut};
\end{tikzpicture}};
\end{tikzpicture}
\fi\fi\par\vspace*{270\p@}}}

%-------------------------------------------

\def\@makeschapterhead#1{%
\begin{tikzpicture}[remember picture,overlay]
\node at (current page.north west)
{\begin{tikzpicture}[remember picture,overlay]
\node[anchor=north west,inner sep=0pt] at (0,0) {\ifusechapterimage\includegraphics[width=\paperwidth]{\thechapterimage}\fi};
\draw[anchor=west] (\Gm@lmargin,-9cm) node [line width=2pt,rounded corners=15pt,draw=ocre,fill=white,fill opacity=0.5,inner sep=15pt]{\strut\makebox[22cm]{}};
\draw[anchor=west] (\Gm@lmargin+.3cm,-9cm) node {\huge\sffamily\bfseries\color{black}#1\strut};
\end{tikzpicture}};
\end{tikzpicture}
\par\vspace*{270\p@}}
\makeatother

%----------------------------------------------------------------------------------------
%	LINKS
%----------------------------------------------------------------------------------------

\usepackage{verbatim}

\usepackage{hyperref}
\hypersetup{hidelinks,backref=true,pagebackref=true,hyperindex=true,colorlinks=false,breaklinks=true,urlcolor=ocre,bookmarks=true,bookmarksopen=false}

\usepackage{bookmark}
\bookmarksetup{
open,
numbered,
addtohook={%
\ifnum\bookmarkget{level}=0 % chapter
\bookmarksetup{bold}%
\fi
\ifnum\bookmarkget{level}=-1 % part
\bookmarksetup{color=ocre,bold}%
\fi
}
}
\begin{comment}
\begin{center}
\begin{multicols}\\
If x is \textbf{positive}:\\
\large{$\frac{(5)+1}{(5)-6}\geq\frac{(5)+2}{(5)-4}$}\\
\large{$\frac{6}{-1}\geq\frac{7}{1}$}\\
\large{$-6\geq7$}\\
\columnbreak
\normalsize{If x is \textbf{negative}:}\\
\large{$\frac{5}{5(-2)-5}\leq\frac{2}{(-2)-1}$}\\
\large{$\frac{5}{-30}\leq\frac{2}{-6}$}\\
\large{$\frac{-1}{6}\leq\frac{-1}{3}$}\\
\end{multicols}
\end{center}
\end{comment}

 % Insert the commands.tex file which contains the majority of the structure behind the template

\usepackage{multicol, wrapfig}
\graphicspath{{./Pictures/}}


\hypersetup{pdftitle={ApplicationofDerivativesPartOne}pdfauthor={Joshua Bautista}}

%----------------------------------------------------------------------------------------

\begin{document}

%----------------------------------------------------------------------------------------
%   TITLE PAGE
%----------------------------------------------------------------------------------------
\part{Application of Derivatives Part One\\ by: Joshua Bautista}
%----------------------------------------------------------------------------------------
%	CHAPTER 1
%----------------------------------------------------------------------------------------

\chapterimage{VelAccelPic.JPG} % Chapter heading image

\pagebreak

\chapter{Application of Derivatives Part One}

\section{Sketching}\index{Sketching}

It is important to sketch out the function in the question to reveal all of its qualities (increasing/decreasing intervals, concave up/down, inflection points, etc).
There is an algorithm to determine all of the details of the graph.

\vspace*{3mm}

\begin{enumerate}
    \item From the original graph:
          \begin{itemize}
              \item You must first \textbf{factor} to check if any \textbf{holes} are in the graph.
              \item State \textbf{VA's} and \textbf{Domain}.
              \item Find the \textbf{x and y intercepts}.
              \item Find the \textbf{end behaviour}.
              \item Look at the behaviour near \textbf{zeros} (x-intercepts) and \textbf{VA's}. (Remember - do this by looking at multiplicities of zeros)
              \item *CAN BE SKIPPED* - Find \textbf{positive and negative intervals} between zeros and VA's.
          \end{itemize}

          \vspace*{4mm}

    \item From the first derivative:
          \begin{itemize}
              \item Find \textbf{critical points}.
              \item *CAN BE SKIPPED* - Find \textbf{increasing/decreasing intervals}.
          \end{itemize}

          \vspace*{4mm}

    \item From the second derivative:
          \begin{itemize}
              \item Find possible \textbf{inflection points}.
              \item Find \textbf{concave up/down intervals}.
              \item Decide if the possible inflection points found are actual inflection points and classify the critical points using the 2nd derivative test.
          \end{itemize}
\end{enumerate}

\pagebreak

\noindent {\large \textbf{Example 1:}}

\vspace*{2mm}

\noindent \emph{Sketch and label all intercepts, asymtotes, crticial points and inflection points. Show all justifying steps}.

\vspace*{2mm}

{\large $y=(x^\frac{1}{3})(x-4)$}

\vspace*{3mm}

\noindent \textbf{1.} Cannot be factored any further. \textbf{$\therefore$ no holes, or VA's. Domain $x \in \Re$} \\

\vspace*{-4mm}

\noindent \textbf{1.} Find x and y intercepts:

\vspace*{-2mm}

\begin{multicols}{2}
    \begin{center}
        \underline{x-intercept} \\
        $0=(x^\frac{1}{3})(x-4)$\\
        by property of zeros: \\
        \textbf{$\therefore (0, 0)$ and $(4, 0)$}

        \columnbreak

        \underline{y-intercept} \\
        $y=((0)^\frac{1}{3})((0)-4)$ \\
        $y=(0)(-4)$ \\
        $y=0$ \\
        \textbf{$\therefore (0, 0)$}
        \columnbreak
    \end{center}
\end{multicols}

\noindent \textbf{1.} Find end behaviour: The function is not rational/exponential. $\therefore$ no HA/OA.

\vspace*{2mm}

\noindent \textbf{2.} Find critical points: Find $y'$ (first derivative) and find when it is equal to 0 or DNE.

\vspace*{-2mm}

\begin{center}
    $y = (x^\frac{1}{3})(x-4)$ \\
    \vspace*{2mm}
    $y' = (x^\frac{1}{3})(1) + (x-4)(\frac{1}{3})(x^\frac{-2}{3})$ \\
    \vspace*{1mm}
    $= (x^\frac{-2}{3})[x+\frac{1}{3}(x-4)]$ \\
    \vspace*{1mm}
    $= (x^\frac{-2}{3})[x+\frac{(x-4)}{3}]$ \\
    \vspace*{1mm}
    $= (x^\frac{-2}{3})(\frac{3x+x-4}{3})$ \\
    \vspace*{1mm}
    {\large $= \frac{4(x-1)}{3x^\frac{2}{3}}$}
    \vspace*{1mm}
\end{center}

\begin{multicols}{2}
    \begin{center}
        $0= \frac{4(x-1)}{3x^\frac{2}{3}}$ \\
        \vspace*{1mm}
        $0 = 4(x-1)$ \\
        \vspace*{1mm}
        $\therefore$ critical pt. $x = 1$
        \columnbreak

        DNE when denominator = 0 \\
        \vspace*{1mm}
        $0 = 3x^\frac{2}{3}$ \\
        \vspace*{1mm}
        $\therefore$ critical pt. $x = 0$
    \end{center}
\end{multicols}

\noindent \textbf{3.} Find possible inflection points: Find $y''$ and find when it is equal to 0 or DNE.

\vspace*{-2mm}

\begin{center}
    $y' = x^\frac{1}{3} + \frac{1}{3}(x)^\frac{-2}{3}(x-4)$ \\
    \vspace*{2mm}
    $y'' = \frac{1}{3}(x)^\frac{-2}{3} + [\frac{-2}{9}(x)^\frac{-5}{3}(x-4)+(\frac{1}{3}(x)^\frac{-2}{3})(1)]$ \\
    \vspace*{1mm}
    {\large $= \frac{1}{3x^\frac{2}{3}}+ (\frac{-2(x-4)}{9x^\frac{5}{3}}+\frac{1}{3x^\frac{2}{3}})$} \\
    \vspace*{1mm}
    {\large $= \frac{3x}{9x^\frac{5}{3}} + (\frac{-2x+8 + 3x}{9x^\frac{5}{3}})$} \\
    \vspace*{1mm}
    {\large $= \frac{4(x+2)}{9x^\frac{5}{3}}$}
\end{center}

\begin{multicols}{2}
    \begin{center}
        $0= \frac{4(x+2)}{9x^\frac{5}{3}}$ \\
        \vspace*{1mm}
        $0 = 4(x+2)$ \\
        \vspace*{1mm}
        $\therefore$ possible inflection pt. $x = -2$
        \columnbreak

        DNE when denominator = 0 \\
        \vspace*{1mm}
        $0 = 9x^\frac{5}{3}$ \\
        \vspace*{1mm}
        $\therefore$ possible inflection pt. $x = 0$
    \end{center}
\end{multicols}

\pagebreak

\noindent \textbf{3.} Find concave up/down intervals: Make a $y''$ sign chart.

\begin{center}
    \begin{table}[h]
        \centering
        \begin{tabular}{l l l l l l}
            \toprule
            \textbf{}        & \textbf{$x<-2$\hspace{3mm}|} & \textbf{$-2<x<0$\hspace{5mm}|} & \textbf{$x>0$\hspace{5mm}|} \\
            \midrule
            $4(x+2)$         & \hspace{5mm}$-$              & \hspace{10mm}$+$               & \hspace{5mm}$+$             \\
            $9x^\frac{5}{3}$ & \hspace{5mm}$-$              & \hspace{10mm}$-$               & \hspace{5mm}$+$             \\
            $y''$            & \hspace{5mm}$+$              & \hspace{10mm}$-$               & \hspace{5mm}$+$             \\
            $y$              & \hspace{5mm}CU               & \hspace{10mm}CD                & \hspace{5mm}CU              \\
            \bottomrule
        \end{tabular}
    \end{table}
\end{center}

\noindent When looking at a sign chart like this, you should look at where the function goes from \textbf{concave up} to \textbf{concave down}. Remember,
a \textbf{positive interval} in the second derivative means it is concave up. A \textbf{negative interval} means it is concave down. There is a CU/CD switch at
$x=-2$ and $x=0$. These are your \textbf{inflection points.} \\

\noindent \textbf{3.} Classify your critical points using the 2nd derivative test.

\begin{multicols}{3}
    \begin{center}
        \underline{$x=-2$} \\
        \vspace*{1mm}
        $y''(-2)=0$ \\
        $\therefore (-2, 7.56)$ \textbf{inflection pt}.
        \columnbreak

        \underline{$x=0$} \\
        \vspace*{1mm}
        $y''(0)=$ DNE, inflection pt. \\
        $\therefore (0, 0)$ is a \textbf{V.T}
        \columnbreak

        \underline{$x=1$} \\
        \vspace*{1mm}
        $y''(1)=\frac{4}{3}$ \\
        CU and $y'(1)=0$ \\
        $\therefore (1, -3)$ \textbf{local min T.P}
    \end{center}
\end{multicols}

\noindent Keep track of all of the classified points (intercepts, critical points, etc) and sketch!

\begin{center}
    \includegraphics[scale=0.15]{Example1Pic.jpg}
\end{center}

\pagebreak

%----------------------------------------------------------------------------------------

\section{Velocity and Acceleration}\index{Notation}

\noindent Before we start solving problems involving velocity and acceleration, we must make sure we know these key definitions:

\vspace*{2mm}

\noindent{\large\textbf{Displacement}} \\
Displacement is the \textbf{change in position} of an object. It is concerned with the initial position of an object to its final position.

\begin{center}
    {\large $Displacement = s(t)$}

\end{center}

\vspace*{2mm}

\noindent{\large\textbf{Velocity}} \\
Velocity is concerned with how fast or slow an objects moves as time changes. It is the derivative of $s(t)$.

\begin{center}
    {\large $Velocity = v(t) = \frac{ds}{dt} = \frac{\Delta s}{\Delta t}$}

\end{center}

\vspace*{2mm}

\noindent{\large\textbf{Acceleration}} \\
Acceleration describes how an object is speeding up or speeding down as time changes. It is the derivative of $v(t)$. (also 2nd derivative of $s(t)$)

\begin{center}
    negative acceleration $\to$ decreasing velocity. \\
    position acceleration $\to$ increasing velocity.
\end{center}

\begin{center}
    {\large $Acceleration = a(t) = \frac{d^2v}{dt^2}$}

\end{center}

\vspace*{2mm}

\noindent{\large\textbf{Jerk/Turbulence}} \\
Jerk/Turbulence is the derivative of $a(t)$.

\begin{center}
    {\large $Jerk/Turbulence = j(t) = \frac{da}{dt} = \frac{d^2v}{dt^2} = \frac{d^3s}{dt^3}$}
\end{center}

\vspace*{2mm}

\begin{center}
    The relationship between displacement, velocity, and acceleration: \\
    \vspace*{2mm}
    {\large $s''(t) = v'(t) = a(t)$}
\end{center}

\noindent{\large\textbf{Graphs}} \\
When looking at displacement, velocity, acceleration, and jerk graphs, there are important things to note.

\vspace*{2mm}

\includegraphics[scale=0.4]{VelAccelPic.JPG}

\pagebreak

\begin{multicols}{2}
    \includegraphics[scale=0.2]{displacementpng.png}
    \columnbreak

    \indent When looking at a displacement-time
    \indent graph, it is important to note that the
    \hspace*{5.25mm} y-values represent the displacement from
    \indent a \textbf{reference point}. The reference point
    \indent is when $time=0$. Positive displacement
    \indent means the object is north or right of the
    \indent reference point.

    \vspace*{3mm}

    \indent To look at velocity, you must take the
    \indent \textbf{slope/derivative} ($\frac{m}{s}$).
\end{multicols}

\begin{multicols}{2}
    \includegraphics[scale=0.8]{velocity.jpg}
    \columnbreak

    \indent When looking at velocity-time graphs,
    \indent the y-values strictly represent the
    \indent object's speed. A $0$ does \textbf{not} mean the
    \indent object is at the origin. A positive velocity
    \indent means the object is moving north, and
    \indent vice-versa. Whenever the graph crosses
    \indent the x-axis, this means it has changed
    \indent directions.

    \vspace*{3mm}

    \indent To look at acceleration, simply take
    \indent the \textbf{slope/derivative} ($\frac{m}{s^2}$).
\end{multicols}

\begin{multicols}{2}
    \includegraphics[scale=0.7]{acceleration.jpg}
    \columnbreak

    \indent When looking at acceleration-time
    \indent graphs, it is key to remember that it is
    \hspace*{2.4mm} the 'rate of change of the rate of change
    \indent of displacement'. So you can visualize
    \indent it as how is the velocity changing as time
    \indent changes.

    \vspace*{3mm}

    \indent Jerk/Turbulence is \textbf{slope/derivative} of
    \indent acceleration.
    \indent
\end{multicols}

\pagebreak

\noindent {\large \textbf{Examples Given Graph: 'What is happening?'}}

\begin{multicols}{2}
    \includegraphics[scale=0.23]{displacementpng.png}
    \columnbreak

    \begin{enumerate}[label=(\alph*)]
        \item The object is moving \textbf{south/left} (away) from the origin (it is becoming negative).
        \item The object is moving back towards the origin (reference point). It goes back to the origin (it crosses the x-axis). It then moves
              \textbf{north/right} from the origin (it is becoming positive).
        \item The object stays \textbf{still/not moving} (constant at +4m).
        \item The object moves back towards the origin and is at the origin again (crosses x-axis).
    \end{enumerate}

\end{multicols}

\begin{multicols}{2}
    \includegraphics[scale=0.8]{velocity.jpg}
    \columnbreak

    \begin{enumerate}[label=(\alph*)]
        \item The object is slowing down going right until it switches direction (crosses x-axis). After switching,
              the object speeds up going left.
        \item The object starts slowing down going left, until it switches direction again. After switching, it speeds up going right.
        \item The object speeds up going right.
        \item The object stays at a constant speed (zero slope) going right.
    \end{enumerate}

\end{multicols}

\begin{multicols}{2}
    \includegraphics[scale=0.8]{acceleration.jpg}
    \columnbreak

    \begin{enumerate}[label=(\alph*)]
        \item The object's velocity is constantly decreasing.
        \item The object's velocity \textbf{drastically} goes from decreasing to increasing by a large number gradually.
        \item The object's velocity is increasing at a constant rate.
        \item The object's velocity \textbf{drastically} goes from increasing to decreasing by a large number gradually.
        \item The object's velocity is decreasing at a constant rate.
    \end{enumerate}

\end{multicols}

\pagebreak

\noindent {\large \textbf{Sketching the Derivative/Anti-Derivative Graphs}}

\begin{multicols}{2}
    \includegraphics[scale=0.23]{displacementpng.png}
    \columnbreak

    

\end{multicols}

%----------------------------------------------------------------------------------------

\section{Other Applications}\index{Operations with Inequalities}

\begin{center}

\end{center}

\vspace*{-3mm}

\pagebreak

\begin{center}

\end{center}

\end{document}
