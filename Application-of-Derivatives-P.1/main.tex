%%%%%%%%%%%%%%%%%%%%%%%%%%%%%%%%%%%%%%%%%
% The Legrand Orange Book
% LaTeX Template
% Version 2.4 (26/09/2018)
%
% This template was downloaded from:
% http://www.LaTeXTemplates.com
%
% Original author:
% Mathias Legrand (legrand.mathias@gmail.com) with modifications by:
% Vel (vel@latextemplates.com)
%
% License:
% CC BY-NC-SA 3.0 (http://creativecommons.org/licenses/by-nc-sa/3.0/)
%
% Compiling this template:
% This template uses biber for its bibliography and makeindex for its index.
% When you first open the template, compile it from the command line with the 
% commands below to make sure your LaTeX distribution is configured correctly:
%
% 1) pdflatex main
% 2) makeindex main.idx -s StyleInd.ist
% 3) biber main
% 4) pdflatex main x 2
%
% After this, when you wish to update the bibliography/index use the appropriate
% command above and make sure to compile with pdflatex several times 
% afterwards to propagate your changes to the document.
%
% This template also uses a number of packages which may need to be
% updated to the newest versions for the template to compile. It is strongly
% recommended you update your LaTeX distribution if you have any
% compilation errors.
%
% Important note:
% Chapter heading images should have a 2:1 width:height ratio,
% e.g. 920px width and 460px height.
%
%%%%%%%%%%%%%%%%%%%%%%%%%%%%%%%%%%%%%%%%%

%----------------------------------------------------------------------------------------
%	PACKAGES AND OTHER DOCUMENT CONFIGURATIONS
%----------------------------------------------------------------------------------------

\documentclass[12pt,fleqn]{book} % Default font size and left-justified equations

\input{structure.tex} % Insert the commands.tex file which contains the majority of the structure behind the template

\usepackage{multicol, wrapfig}
\graphicspath{{./Pictures/}}


\hypersetup{pdftitle={ApplicationofDerivativesPartOne}pdfauthor={Joshua Bautista}}

%----------------------------------------------------------------------------------------

\begin{document}

%----------------------------------------------------------------------------------------
%   TITLE PAGE
%----------------------------------------------------------------------------------------
\part{Application of Derivatives Part One\\ by: Joshua Bautista}
%----------------------------------------------------------------------------------------
%	CHAPTER 1
%----------------------------------------------------------------------------------------

\chapterimage{VelAccelPic.JPG} % Chapter heading image

\pagebreak

\chapter{Application of Derivatives Part One}

\vspace*{-50mm}

\section{Sketching}\index{Sketching}

\vspace*{-70mm}

It is important to sketch out the function in the question to reveal all of its characteristics (increasing/decreasing intervals, concave up/down, inflection points, etc).
There is an algorithm to help sketch a graph given its function/equation:

\vspace*{-35mm}

\begin{enumerate}
    \item From the original graph:
          \begin{itemize}
              \item You must first \textbf{factor} to check if any \textbf{holes} are in the graph.
              \item State \textbf{VA's} and \textbf{Domain}.
              \item Find the \textbf{x and y intercepts}.
              \item Find the \textbf{end behaviour}.
              \item Look at the behaviour near \textbf{zeros} (x-intercepts) and \textbf{VA's}. (Remember - do this by looking at multiplicities of zeros)
              \item *CAN BE SKIPPED* - Find \textbf{positive and negative intervals} between zeros and VA's.
          \end{itemize}

          \vspace*{4mm}

    \item From the first derivative:
          \begin{itemize}
              \item Find \textbf{critical points}.
              \item *CAN BE SKIPPED* - Find \textbf{increasing/decreasing intervals}.
          \end{itemize}

          \vspace*{4mm}

    \item From the second derivative:
          \begin{itemize}
              \item Find possible \textbf{inflection points}.
              \item Find \textbf{concave up/down intervals}.
              \item Decide if the possible inflection points found are actual inflection points and classify the critical points using the 2nd derivative test.
          \end{itemize}
\end{enumerate}

\pagebreak

\noindent {\large \textbf{Example}}

\vspace*{2mm}

\noindent \emph{Sketch and label all intercepts, asymtotes, crticial points and inflection points. Show all justifying steps}.

\vspace*{2mm}

{\large $y=(x^\frac{1}{3})(x-4)$}

\vspace*{3mm}

\noindent \textbf{1.} Cannot be factored any further. \textbf{$\therefore$ no holes, or VA's. Domain $x \in \Re$} \\

\vspace*{-4mm}

\noindent \textbf{1.} Find x and y intercepts:

\vspace*{-2mm}

\begin{multicols}{2}
    \begin{center}
        \underline{x-intercept} \\
        $0=(x^\frac{1}{3})(x-4)$\\
        by property of zeros: \\
        \textbf{$\therefore (0, 0)$ and $(4, 0)$}

        \columnbreak

        \underline{y-intercept} \\
        $y=((0)^\frac{1}{3})((0)-4)$ \\
        $y=(0)(-4)$ \\
        $y=0$ \\
        \textbf{$\therefore (0, 0)$}
        \columnbreak
    \end{center}
\end{multicols}

\noindent \textbf{1.} Find end behaviour: The function is not rational/exponential. $\therefore$ no HA/OA.

\vspace*{2mm}

\noindent \textbf{2.} Find critical points: Find $y'$ (first derivative) and find when it is equal to 0 or DNE.

\vspace*{-2mm}

\begin{center}
    $y = (x^\frac{1}{3})(x-4)$ \\
    \vspace*{2mm}
    $y' = (x^\frac{1}{3})(1) + (x-4)(\frac{1}{3})(x^\frac{-2}{3})$ \\
    \vspace*{1mm}
    $= (x^\frac{-2}{3})[x+\frac{1}{3}(x-4)]$ \\
    \vspace*{1mm}
    $= (x^\frac{-2}{3})[x+\frac{(x-4)}{3}]$ \\
    \vspace*{1mm}
    $= (x^\frac{-2}{3})(\frac{3x+x-4}{3})$ \\
    \vspace*{1mm}
    {\large $= \frac{4(x-1)}{3x^\frac{2}{3}}$}
    \vspace*{1mm}
\end{center}

\begin{multicols}{2}
    \begin{center}
        $0= \frac{4(x-1)}{3x^\frac{2}{3}}$ \\
        \vspace*{1mm}
        $0 = 4(x-1)$ \\
        \vspace*{1mm}
        $\therefore$ critical pt. $x = 1$
        \columnbreak

        DNE when denominator = 0 \\
        \vspace*{1mm}
        $0 = 3x^\frac{2}{3}$ \\
        \vspace*{1mm}
        $\therefore$ critical pt. $x = 0$
    \end{center}
\end{multicols}

\noindent \textbf{3.} Find possible inflection points: Find $y''$ and find when it is equal to 0 or DNE.

\vspace*{-2mm}

\begin{center}
    $y' = x^\frac{1}{3} + \frac{1}{3}(x)^\frac{-2}{3}(x-4)$ \\
    \vspace*{2mm}
    $y'' = \frac{1}{3}(x)^\frac{-2}{3} + [\frac{-2}{9}(x)^\frac{-5}{3}(x-4)+(\frac{1}{3}(x)^\frac{-2}{3})(1)]$ \\
    \vspace*{1mm}
    {\large $= \frac{1}{3x^\frac{2}{3}}+ (\frac{-2(x-4)}{9x^\frac{5}{3}}+\frac{1}{3x^\frac{2}{3}})$} \\
    \vspace*{1mm}
    {\large $= \frac{3x}{9x^\frac{5}{3}} + (\frac{-2x+8 + 3x}{9x^\frac{5}{3}})$} \\
    \vspace*{1mm}
    {\large $= \frac{4(x+2)}{9x^\frac{5}{3}}$}
\end{center}

\begin{multicols}{2}
    \begin{center}
        $0= \frac{4(x+2)}{9x^\frac{5}{3}}$ \\
        \vspace*{1mm}
        $0 = 4(x+2)$ \\
        \vspace*{1mm}
        $\therefore$ possible inflection pt. $x = -2$
        \columnbreak

        DNE when denominator = 0 \\
        \vspace*{1mm}
        $0 = 9x^\frac{5}{3}$ \\
        \vspace*{1mm}
        $\therefore$ possible inflection pt. $x = 0$
    \end{center}
\end{multicols}

\pagebreak

\noindent \textbf{3.} Find concave up/down intervals: Make a $y''$ sign chart.

\begin{center}
    \begin{table}[h]
        \centering
        \begin{tabular}{l l l l l l}
            \toprule
            \textbf{}        & \textbf{$x<-2$\hspace{3mm}|} & \textbf{$-2<x<0$\hspace{5mm}|} & \textbf{$x>0$\hspace{5mm}|} \\
            \midrule
            $4(x+2)$         & \hspace{5mm}$-$              & \hspace{10mm}$+$               & \hspace{5mm}$+$             \\
            $9x^\frac{5}{3}$ & \hspace{5mm}$-$              & \hspace{10mm}$-$               & \hspace{5mm}$+$             \\
            $y''$            & \hspace{5mm}$+$              & \hspace{10mm}$-$               & \hspace{5mm}$+$             \\
            $y$              & \hspace{5mm}CU               & \hspace{10mm}CD                & \hspace{5mm}CU              \\
            \bottomrule
        \end{tabular}
    \end{table}
\end{center}

\noindent When looking at a sign chart like this, you should look at where the function goes from \textbf{concave up} to \textbf{concave down}. Remember,
a \textbf{positive interval} in the second derivative means it is concave up. A \textbf{negative interval} means it is concave down. There is a CU/CD switch at
$x=-2$ and $x=0$. These are your \textbf{inflection points.} \\

\noindent \textbf{3.} Classify your critical points using the 2nd derivative test.

\begin{multicols}{3}
    \begin{center}
        \underline{$x=-2$} \\
        \vspace*{1mm}
        $y''(-2)=0$ \\
        $\therefore (-2, 7.56)$ \textbf{inflection pt}.
        \columnbreak

        \underline{$x=0$} \\
        \vspace*{1mm}
        $y''(0)=$ DNE, inflection pt. \\
        $\therefore (0, 0)$ is a \textbf{V.T}
        \columnbreak

        \underline{$x=1$} \\
        \vspace*{1mm}
        $y''(1)=\frac{4}{3}$ \\
        CU and $y'(1)=0$ \\
        $\therefore (1, -3)$ \textbf{local min T.P}
    \end{center}
\end{multicols}

\noindent Keep track of all of the classified points (intercepts, critical points, etc) and sketch!

\begin{center}
    \includegraphics[scale=0.15]{Example1Pic.jpg}
\end{center}

\pagebreak

%----------------------------------------------------------------------------------------

\section{Velocity and Acceleration}\index{Notation}

\noindent Before we start solving problems involving velocity and acceleration, we must make sure we know these key definitions:

\vspace*{2mm}

\noindent{\large\textbf{Displacement}} \\
Displacement is the \textbf{change in position} of an object. It is concerned with the initial position of an object to its final position.

\begin{center}
    {\large $Displacement = s(t)$}

\end{center}

\vspace*{2mm}

\noindent{\large\textbf{Velocity}} \\
Velocity is concerned with how fast or slow an objects moves as time changes. It is the derivative of $s(t)$.

\begin{center}
    {\large $Velocity = v(t) = \frac{ds}{dt} = \frac{\Delta s}{\Delta t}$}

\end{center}

\vspace*{2mm}

\noindent{\large\textbf{Acceleration}} \\
Acceleration describes how an object is speeding up or speeding down as time changes. It is the derivative of $v(t)$. (also 2nd derivative of $s(t)$)

\begin{center}
    negative acceleration $\to$ decreasing velocity. \\
    position acceleration $\to$ increasing velocity.
\end{center}

\begin{center}
    {\large $Acceleration = a(t) = \frac{d^2v}{dt^2}$}

\end{center}

\vspace*{2mm}

\noindent{\large\textbf{Jerk/Turbulence}} \\
Jerk/Turbulence is the derivative of $a(t)$.

\begin{center}
    {\large $Jerk/Turbulence = j(t) = \frac{da}{dt} = \frac{d^2v}{dt^2} = \frac{d^3s}{dt^3}$}
\end{center}

\vspace*{2mm}

\begin{center}
    The relationship between displacement, velocity, and acceleration: \\
    \vspace*{2mm}
    {\large $s''(t) = v'(t) = a(t)$}
\end{center}

\noindent{\large\textbf{Graphs}} \\
When looking at displacement, velocity, acceleration, and jerk graphs, there are important things to note.

\vspace*{2mm}

\includegraphics[scale=0.4]{VelAccelPic.JPG}

\pagebreak

\begin{multicols}{2}
    \includegraphics[scale=0.2]{displacementpng.png}
    \columnbreak

    \indent When looking at a displacement-time
    \indent graph, it is important to note that the
    \hspace*{5.25mm} y-values represent the displacement from
    \indent a \textbf{reference point}. The reference point
    \indent is when $time=0$. Positive displacement
    \indent means the object is north or right of the
    \indent reference point.

    \vspace*{3mm}

    \indent To look at velocity, you must take the
    \indent \textbf{slope/derivative} ($\frac{m}{s}$).
\end{multicols}

\begin{multicols}{2}
    \includegraphics[scale=0.8]{velocity.jpg}
    \columnbreak

    \indent When looking at velocity-time graphs,
    \indent the y-values strictly represent the
    \indent object's speed. A $0$ does \textbf{not} mean the
    \indent object is at the origin. A positive velocity
    \indent means the object is moving north, and
    \indent vice-versa. Whenever the graph crosses
    \indent the x-axis, this means it has changed
    \indent directions.

    \vspace*{3mm}

    \indent To look at acceleration, simply take
    \indent the \textbf{slope/derivative} ($\frac{m}{s^2}$).
\end{multicols}

\begin{multicols}{2}
    \includegraphics[scale=0.7]{acceleration.jpg}
    \columnbreak

    \indent When looking at acceleration-time
    \indent graphs, it is key to remember that it is
    \hspace*{2.4mm} the 'rate of change of the rate of change
    \indent of displacement'. So you can visualize
    \indent it as how is the velocity changing as time
    \indent changes.

    \vspace*{3mm}

    \indent Jerk/Turbulence is \textbf{slope/derivative} of
    \indent acceleration.
    \indent
\end{multicols}

\pagebreak

\noindent {\large \textbf{Examples Given Graph: 'What is happening?'}}

\begin{multicols}{2}
    \includegraphics[scale=0.23]{displacementpng.png}
    \columnbreak

    \begin{enumerate}[label=(\alph*)]
        \item The object is moving \textbf{south/left} (away) from the origin (it is becoming negative).
        \item The object is moving back towards the origin (reference point). It goes back to the origin (it crosses the x-axis). It then moves
              \textbf{north/right} from the origin (it is becoming positive).
        \item The object stays \textbf{still/not moving} (constant at +4m).
        \item The object moves back towards the origin and is at the origin again (crosses x-axis).
    \end{enumerate}

\end{multicols}

\begin{multicols}{2}
    \includegraphics[scale=0.8]{velocity.jpg}
    \columnbreak

    \begin{enumerate}[label=(\alph*)]
        \item The object is slowing down going right until it switches direction (crosses x-axis). After switching,
              the object speeds up going left.
        \item The object starts slowing down going left, until it switches direction again. After switching, it speeds up going right.
        \item The object speeds up going right.
        \item The object stays at a constant speed (zero slope) going right.
    \end{enumerate}

\end{multicols}

\begin{multicols}{2}
    \includegraphics[scale=0.8]{acceleration.jpg}
    \columnbreak

    \begin{enumerate}[label=(\alph*)]
        \item The object's velocity is constantly decreasing.
        \item The object's velocity \textbf{instantly} goes from decreasing to increasing by a large number constantly.
        \item The object's velocity is increasing at a constant rate.
        \item The object's velocity \textbf{instantly} goes from increasing to decreasing by a large number constantly.
        \item The object's velocity is decreasing at a constant rate.
    \end{enumerate}

\end{multicols}

\pagebreak

\noindent {\large \textbf{Sketching the Derivative/Anti-Derivative Graphs}}

\begin{multicols}{3}
    \hspace*{-10mm}
    \includegraphics[scale=0.2]{displacementpng.png}
    \columnbreak
    \hspace*{10mm}
    \includegraphics[scale=0.06]{displace-vel.jpg}

    \columnbreak
    \begin{enumerate}[label=(\alph*)]
        \item Constant velocity of $-2\frac{m}{s}$. \hspace*{-3mm} Object going left.
        \item Constant velocity of $5\frac{m}{s}$. \hspace*{-2mm} Object going right.
        \item Constant velocity of $0\frac{m}{s}$. \hspace*{-3mm} Object staying still.
        \item Constant velocity of $-5\frac{m}{s}$. \hspace*{-3mm} Object going left.
    \end{enumerate}

\end{multicols}


\begin{multicols}{2}
    \includegraphics[scale=0.7]{velocity.jpg} \\
    (zeros become t.p and t.p become inflection pt) \\
    \vspace*{-4mm}
    \begin{center}
        \includegraphics[scale=0.05]{vel-accel.jpg}
        \begin{enumerate}[label=(\alph*)]
            \item Constant negative acceleration directed to the left (decreasing velocity).
            \item Constant positive acceleration directed to the right (increasing velocity).
            \item Larger constant positive acceleration directed to the right.
            \item 0 acceleration (constant velocity).
        \end{enumerate}

    \end{center}
    \columnbreak

    \hspace*{-20mm}

    \begin{center}
        \vspace*{3mm}
        \includegraphics[scale=0.05]{vel-displace.jpg}
        \begin{enumerate}[label=(\alph*)]
            \item NOTE: We do not know where the object starts. The object first moves away (right) from the origin, then switches direction and moves toward (left) the origin speeding up.
            \item The object slows down, still moving away from the origin, until it switches direction once again.
            \item Object speeds up while moving toward (right) the origin.
            \item Object is moving at a constant rate away (right) from the origin.
        \end{enumerate}

    \end{center}
\end{multicols}

\pagebreak

\begin{multicols}{2}
    \includegraphics[scale=0.7]{acceleration.jpg} \\
    \begin{center}

        \includegraphics[scale=0.05]{accel-vel.jpg}
        \begin{enumerate}[label=(\alph*)]
            \item Object is speeding up at a constant acceleration going away (left) from the origin (negative acceleration).
            \item Object speeds up instantly, now at a positive acceleration.
            \item Object is speeding up at a constant acceleration going toward the origin (positive acceleration). It passes origin (in this case) and continues  will constant velocity away (right) from the origin.
            \item Objects slows down instantly, now at a negative acceleration.
            \item Object is speeding up at a constant acceleration going away (left) from the origin (negative acceleration).
        \end{enumerate}

    \end{center}
    \columnbreak

    \hspace*{-20mm}

    \begin{center}
        \vspace*{10mm}
        \includegraphics[scale=0.05]{accel-jerk.jpg} \\
        The derivative or 'rate of change' of acceleration in this case are all constant 0 (no slope in original acceleration graph), except the points where the acceleration instantly increases/decreases.

    \end{center}
\end{multicols}

\pagebreak

%----------------------------------------------------------------------------------------

\section{Other Applications}\index{Operations with Inequalities}

\includegraphics[scale=0.85]{apply1.jpg}

\noindent \textbf{a)} \\
We are given a function that gives the height of the ball after $t$ seconds. To get velocity, we must get the \textbf{slope/derivative} at 2 seconds. Let's get the derivative of the given function:

\begin{center}
    {\large $y=f(t)=-4.9t^2+10t$} \\
    {\large $f'(t)=-9.8t + 10$}
\end{center}

\noindent Plug-in 2 seconds into the derivative.

\begin{center}
    {\large $f'(2)=-9.8(2)+10$} \\
    {\large $=-19.6+10$} \\
    {\large $=-9.6$} \\
    $\therefore$ velocity $-9.6\frac{m}{s}$ at 2 seconds.
\end{center}

\noindent \textbf{b)} \\
At its highest point, the derivative will be equal to $0$. We can set the derivative function equal to 0 and find at what second it is 0.

\begin{center}
    {\large $-9.8t+10 =0$} \\
    {\large $-9.8t=-10$} \\
    {\large $t=\frac{10}{9.8}$} \\
    {\large $t=1.020408163$} \\
    $\therefore$ at 1.02 seconds the ball is at highest point
\end{center}

\noindent \textbf{c)} \\
When the ball is on the ground, its height will be equal to 0. We can set the original function equal to 0 and solve for $t$.

\begin{center}
    {\large $-4.9t^2+10t=0$} \\
    \vspace*{1mm}
    {\large $t=\frac{-10\pm\sqrt{10^2-4(-4.9)(0)}}{2(-4.9)}$} \\
    \vspace*{1mm}
    {\large $t=\frac{-10\pm\sqrt{100}}{-9.8}$}
    \begin{multicols}{2}
        {\large $t=\frac{-20}{-9.8}$} \\
        $t=2.04081$

        \columnbreak

        {\large $t=\frac{0}{-9.8}$} \\
        $t=0$
    \end{multicols}
\end{center}

\noindent $\therefore$ The ball will be back on the ground at 2.04 seconds.

\pagebreak

\includegraphics[scale=0.85]{apply2.jpg}

\noindent \textbf{a)} \\
The derivative is also the slope, the units you get when calculating the slope is $\frac{P}{n}$.

\begin{center}
    {\large $\frac{\Delta P}{\Delta n} = \frac{items completed}{person}$}
\end{center}

\noindent \textbf{b)} \\
$\frac{\Delta P}{\Delta n}$ refers to the number of items completed as the number of people working on the job changes.

\vspace*{3mm}

\noindent \textbf{c)} \\
Using logic, we can say that more items should be completed as the number of people working on the job increase. $\therefore \frac{\Delta P}{\Delta n} > 0$

\vspace*{3mm}

\noindent \textbf{d)} \\
Given $\frac{\Delta P}{\Delta n} = 300$ when $n=6$, we can say that adding one more person working on the job when there are already 6 people working, will increase the number of items completed by 300.


\end{document}
